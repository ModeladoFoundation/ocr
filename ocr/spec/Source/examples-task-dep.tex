%%%%
\subsection{Code example}
This example illustrates several aspect of the OCR API with regards to
the creation of an irregular task graph. Specifically, it illustrates:
\begin{enumerate}
\item{Adding dependences between {\bf a)} events and EDTs, {\bf b)}
    data-blocks and EDTs, and {\bf c)} the NULL\_GUID and EDTs;}
\item{The use of an EDT's post-slot and how a ``producer'' EDT can
    pass a data-block to a ``consumer'' EDT using this post-slot;}
\item{Several methods of satisfying an EDT's pre-slot: {\bf a)}
    through the use of an explicit dependence array at creation time,
    {\bf b)} through the use of another EDT's post-slot and {\bf c)}
    through the use of an explicitly added dependence followed by an
    \texttt{ocrEventSatisfy} call.}
\end{enumerate}
\begin{ocrsnip}
/* Example of a pattern that highlights the
 * expressiveness of task dependences
 *
 * Implements the following dependence graph:
 * (@ \label{line:task-dep-graph} @)
 * mainEdt
 * |      \
 * stage1a stage1b
 * |     \       |
 * |      \      |
 * |       \     |
 * stage2a  stage2b
 *     \      /
 *     shutdownEdt
 */
#include "ocr.h"

#define NB_ELEM_DB 20

ocrGuid_t shutdownEdt(u32 paramc, u64* paramv, u32 depc, ocrEdtDep_t depv[]) {
    ASSERT(depc == 2);
    u64* data0 = (u64*)depv[0].ptr;
    u64* data1 = (u64*)depv[1].ptr;

    ASSERT(*data0 == 3ULL);
    ASSERT(*data1 == 4ULL);
    PRINTF("Got a DB (GUID 0x%lx) containing %lu on slot 0\n", depv[0].guid, *data0);
    PRINTF("Got a DB (GUID 0x%lx) containing %lu on slot 1\n", depv[1].guid, *data1);

    // Free the data-blocks that were passed in
    ocrDbDestroy(depv[0].guid);
    ocrDbDestroy(depv[1].guid);

    // Shutdown the runtime
    ocrShutdown();
    return NULL_GUID;
}

ocrGuid_t stage2a(u32 paramc, u64* paramv, u32 depc, ocrEdtDep_t depv[]);

ocrGuid_t stage1a(u32 paramc, u64* paramv, u32 depc, ocrEdtDep_t depv[]) {
    ASSERT(depc == 1);
    ASSERT(paramc == 1);
    // paramv[0] is the event that the child EDT has to satisfy
    // when it is done

    // We create a data-block for one u64 and put data in it
    ocrGuid_t dbGuid = NULL_GUID, stage2aTemplateGuid = NULL_GUID,
        stage2aEdtGuid = NULL_GUID;
    u64* dbPtr = NULL;
    ocrDbCreate(&dbGuid, (void**)&dbPtr, sizeof(u64), 0, NULL_GUID, NO_ALLOC);
    *dbPtr = 1ULL;

    // Create an EDT and pass it the data-block we just created
    // The EDT is immediately ready to execute
    ocrEdtTemplateCreate(&stage2aTemplateGuid, stage2a, 1, 1);
    ocrEdtCreate(&stage2aEdtGuid, stage2aTemplateGuid, EDT_PARAM_DEF,(@ \label{line:expDep} @)
                 paramv, EDT_PARAM_DEF, &dbGuid, EDT_PROP_NONE, NULL_GUID, NULL);

    // Pass the same data-block created to stage2b (links setup in mainEdt)
    return dbGuid; (@ \label{line:EdtGuidReturn} @)
}

ocrGuid_t stage1b(u32 paramc, u64* paramv, u32 depc, ocrEdtDep_t depv[]) {
    ASSERT(depc == 1);
    ASSERT(paramc == 0);

    // We create a data-block for one u64 and put data in it
    ocrGuid_t dbGuid = NULL_GUID;
    u64* dbPtr = NULL;
    ocrDbCreate(&dbGuid, (void**)&dbPtr, sizeof(u64), 0, NULL_GUID, NO_ALLOC);
    *dbPtr = 2ULL;

    // Pass the created data-block created to stage2b (links setup in mainEdt)
    return dbGuid;
}

ocrGuid_t stage2a(u32 paramc, u64* paramv, u32 depc, ocrEdtDep_t depv[]) {
    ASSERT(depc == 1);
    ASSERT(paramc == 1);

    u64 *dbPtr = (u64*)depv[0].ptr;
    ASSERT(*dbPtr == 1ULL); // We got this from stage1a

    *dbPtr = 3ULL; // Update the value

    // Pass the modified data-block to shutdown
    ocrEventSatisfy((ocrGuid_t)paramv[0], depv[0].guid); (@ \label{line:expSatisfy} @)

    return NULL_GUID;
}

ocrGuid_t stage2b(u32 paramc, u64* paramv, u32 depc, ocrEdtDep_t depv[]) {
    ASSERT(depc == 2);
    ASSERT(paramc == 0);

    u64 *dbPtr = (u64*)depv[1].ptr;
    // Here, we can run concurrently to stage2a which modifies the value
    // we see in depv[0].ptr. We should see either 1ULL or 3ULL

    // On depv[1], we get the value from stage1b and it should be 2
    ASSERT(*dbPtr == 2ULL); // We got this from stage2a

    *dbPtr = 4ULL; // Update the value

    return depv[1].guid; // Pass this to the shudown EDT
}


ocrGuid_t mainEdt(u32 paramc, u64* paramv, u32 depc, ocrEdtDep_t depv[]) {

    // Create the shutdown EDT
    ocrGuid_t stage1aTemplateGuid = NULL_GUID, stage1bTemplateGuid = NULL_GUID,
        stage2bTemplateGuid = NULL_GUID, shutdownEdtTemplateGuid = NULL_GUID;
    ocrGuid_t shutdownEdtGuid = NULL_GUID, stage1aEdtGuid = NULL_GUID,
        stage1bEdtGuid = NULL_GUID, stage2bEdtGuid = NULL_GUID,
        evtGuid = NULL_GUID, stage1aOut = NULL_GUID, stage1bOut = NULL_GUID,
        stage2bOut = NULL_GUID;

    ocrEdtTemplateCreate(&shutdownEdtTemplateGuid, shutdownEdt, 0, 2);
    ocrEdtCreate(&shutdownEdtGuid, shutdownEdtTemplateGuid, 0, NULL, EDT_PARAM_DEF, NULL,
                 EDT_PROP_NONE, NULL_GUID, NULL);

    // Create the event to satisfy shutdownEdt by stage 2a
    // (stage 2a is created by 1a)
    ocrEventCreate(&evtGuid, OCR_EVENT_ONCE_T, true);

    // Create stages 1a, 1b and 2b
    // For 1a and 1b, add a "fake" dependence to avoid races between
    // setting up the event links and running the EDT
    ocrEdtTemplateCreate(&stage1aTemplateGuid, stage1a, 1, 1);
    ocrEdtCreate(&stage1aEdtGuid, stage1aTemplateGuid, EDT_PARAM_DEF, &evtGuid,
                 EDT_PARAM_DEF, NULL, EDT_PROP_NONE, NULL_GUID, &stage1aOut);

    ocrEdtTemplateCreate(&stage1bTemplateGuid, stage1b, 0, 1);
    ocrEdtCreate(&stage1bEdtGuid, stage1bTemplateGuid, EDT_PARAM_DEF, NULL,
                 EDT_PARAM_DEF, NULL, EDT_PROP_NONE, NULL_GUID, &stage1bOut);

    ocrEdtTemplateCreate(&stage2bTemplateGuid, stage2b, 0, 2);
    ocrEdtCreate(&stage2bEdtGuid, stage2bTemplateGuid, EDT_PARAM_DEF, NULL,
                 EDT_PARAM_DEF, NULL, EDT_PROP_NONE, NULL_GUID, &stage2bOut);

    // Set up all the links
    // 1a -> 2b
    ocrAddDependence(stage1aOut, stage2bEdtGuid, 0, DB_DEFAULT_MODE); (@ \label{line:outEvtToEdtDep} @)

    // 1b -> 2b
    ocrAddDependence(stage1bOut, stage2bEdtGuid, 1, DB_DEFAULT_MODE);

    // Event satisfied by 2a -> shutdown
    ocrAddDependence(evtGuid, shutdownEdtGuid, 0, DB_DEFAULT_MODE);
    // 2b -> shutdown
    ocrAddDependence(stage2bOut, shutdownEdtGuid, 1, DB_DEFAULT_MODE);

    // Start 1a and 1b
    ocrAddDependence(NULL_GUID, stage1aEdtGuid, 0, DB_DEFAULT_MODE); (@ \label{line:nullGuidToEdt} @)
    ocrAddDependence(NULL_GUID, stage1bEdtGuid, 0, DB_DEFAULT_MODE);

    return NULL_GUID;
}
\end{ocrsnip}
%%%
\subsubsection{Details}
%%%%
%\paragraph{The graph construction}
The snippet of code shows one possible way to construct the irregular
task-graph shown starting on
Line~\ref{line:task-dep-graph}. \texttt{mainEdt} will create {\bf a)}
\texttt{stage1a} and \texttt{stage1b} as they are the next things that
need to execute but also {\bf b)} \texttt{stage2b} and
\texttt{shutdownEdt} because it is the immediate dominator of those
EDTs. In general, it is easiest to create an EDT in its immediate
dominator because that allows any other EDTs who need to feed it
information (necessarily between its dominator and the EDT in
question) to be able to know the value of the opaque GUID created for
hte EDT. \texttt{stage2a}, on the other hand, can be created by
\texttt{stage1a} as no-one else needs to feed information to it.

Most of the ``edges'' in the dependence graph are also created in
\texttt{mainEdt} starting at Line~\ref{line:outEvtToEdtDep}. These
are either between the post-slot (output event) of a source EDT and
an EDT or between a regular event and an EDT. Note also the use of
NULL\_GUID as a source for two dependences starting at
Line~\ref{line:nullGuidToEdt}. A NULL\_GUID as a source for a
dependence immediately satisfies the destination slot; in this case,
it satisfies the unique dependence of \texttt{stage1a} and
\texttt{stage1b} and makes them runable. These two dependences do not
exist in the graph shown starting at Line~\ref{line:task-dep-graph}
but are crucial to avoid a potential race in the program: the output
events of EDTs are similar to ONCE events in the sense that they will
disappear once they are satisfied and therefore, any dependence on
them must be properly setup prior to their potential satisfaction. In
other words, the \texttt{ocrAddDependence} calls starting at
Line~\ref{line:outEvtToEdtDep} must \emph{happen-before} the
satisfaction of \texttt{stage1aOut} and \texttt{stage1bOut}.
%%%%
%\paragraph{Satisfying pre-slots}
This example shows three methods of satisfying an EDT's pre-slots:
\begin{itemize}
\item{Through the use of an explicit dependence array known at EDT
    creation time as shown on Line~\ref{line:expDep};}
\item{Through an output event as shown on
    Line~\ref{line:EdtGuidReturn}. The GUID passed as a return value
    of the EDT function will be passed to the EDT's output event (in
    this case \texttt{stage1aOut}). If the GUID is a data-block's
    GUID, the output event will be satisfied with that data-block. If
    it is an event's GUID, the two events will become linked;}
\item{Through an explicit satisfaction as shown on
    Line~\ref{line:expSatisfy}).}
\end{itemize}
