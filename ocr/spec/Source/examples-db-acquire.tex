This example illustrates the usage model for accessing the contents of a data block.
The data contents of a data block are made available to the EDT through the input slots in depv.
The input slots contain two fields: the GUID of the data block and pointer to the contents of the data block.
The runtime process grabs a pointer to the contents through a process called ``acquire''.
The acquires of all data blocks accessed inside the EDT have to happen
before the EDT starts execution. This implies that the runtime requires knowledge of
which data blocks it needs to acquire. That information is given to the runtime through
the process of dependence satisfaction. As a result, a data block's contents are available
to the EDT only if that data block has been passed in as the input on a dependence slot or
if the data block is created inside the EDT.
%%%%
\subsection{Code example}
\begin{ocrsnip}
/* Example to show how DB guids can be passed through another DB.
 * Note: DB contents can be accessed by an EDT only when they arrive
 * in a dependence slot.
 *
 * Implements the following dependence graph:
 *
 *     mainEdt
 *     [ DB1 ]
 *        |
 *       EDT1
 *        |
 *     [ DB0 ]
 *   shutdownEdt
 *
 */

#include "ocr.h"

#define VAL 42

ocrGuid_t exampleEdt(u32 paramc, u64* paramv, u32 depc, ocrEdtDep_t depv[]) {
    ocrGuid_t *dbPtr = (ocrGuid_t*)depv[0].ptr;
    ocrGuid_t passedDb = dbPtr[0];
    PRINTF("Passing DB: "GUIDF"\n", GUIDA(passedDb));
    ocrDbDestroy(depv[0].guid);
    return passedDb;
}

ocrGuid_t awaitingEdt(u32 paramc, u64* paramv, u32 depc, ocrEdtDep_t depv[]) {
    u64 i;
    u32 *dbPtr = (u32*)depv[0].ptr;
    PRINTF("Received: %"PRIu32"\n", dbPtr[0]);
    ocrDbDestroy(depv[0].guid);
    ocrShutdown();
    return NULL_GUID;
}

ocrGuid_t mainEdt(u32 paramc, u64* paramv, u32 depc, ocrEdtDep_t depv[]) {
    u32 i;

    // Create DBs
    u32* ptr0;
    ocrGuid_t* ptr1;
    ocrGuid_t db0Guid, db1Guid;
    ocrDbCreate(&db0Guid, (void**)&ptr0, sizeof(u32), DB_PROP_NONE, NULL_HINT, NO_ALLOC);
    ocrDbCreate(&db1Guid, (void**)&ptr1, sizeof(ocrGuid_t), DB_PROP_NONE, NULL_HINT, NO_ALLOC);
    ptr0[0] = VAL;
    ptr1[0] = db0Guid;
    PRINTF("Sending: %"PRIu32" in DB: "GUIDF"\n", ptr0[0], GUIDA(db0Guid));
    ocrDbRelease(db0Guid);
    ocrDbRelease(db1Guid);

    // Create Middle EDT
    ocrGuid_t exampleTemplGuid, exampleEdtGuid, exampleEventGuid;
    ocrEdtTemplateCreate(&exampleTemplGuid, exampleEdt, 0 /*paramc*/, 1 /*depc*/);
    ocrEdtCreate(&exampleEdtGuid, exampleTemplGuid, EDT_PARAM_DEF, NULL, EDT_PARAM_DEF, NULL,
        EDT_PROP_NONE, NULL_HINT, &exampleEventGuid);

    // Create AWAIT EDT
    ocrGuid_t awaitingTemplGuid, awaitingEdtGuid;
    ocrEdtTemplateCreate(&awaitingTemplGuid, awaitingEdt, 0 /*paramc*/, 1 /*depc*/);
    ocrEdtCreate(&awaitingEdtGuid, awaitingTemplGuid, EDT_PARAM_DEF, NULL, EDT_PARAM_DEF, NULL,
        EDT_PROP_NONE, NULL_HINT, NULL);
    ocrAddDependence(exampleEventGuid, awaitingEdtGuid, 0, DB_DEFAULT_MODE);

    // START Middle EDT
    ocrAddDependence(db1Guid, exampleEdtGuid, 0, DB_DEFAULT_MODE);

    return NULL_GUID;
}
\end{ocrsnip}
%%%
\subsubsection{Details}

The mainEdt creates two data blocks ({\tt db0Guid} and {\tt db1Guid}).
It then sets the content of {\tt db0Guid} to be an user-define value,
while the content of {\tt db1Guid} is set to be the GUID value of {\tt db0Guid}.
The runtime then creates an EDT ({\tt exampleEdtGuid}) that takes one input dependence.
It creates another EDT ({\tt awaitingEdtGuid}) and makes it dependent on the satisfaction of the {\tt exampleEdtGuid}'s output event ({\tt exampleEventGuid}).
Finally, mainEdt satisfies the dependence of {\tt exampleEdtGuid} with the data block {\tt db1Guid}.

Once {\tt exampleEdtGuid} starts executing function ``exampleEdt'', the contents of {\tt db1Guid} are read.
The function then retrieves the GUID of the data block {\tt db0Guid} from the contents of {\tt db1Guid}.
Now in order to read the contents of {\tt db0Guid}, the function satisfies the output event with {\tt db0Guid}.

Inside the final EDT function ``awaitingEdt'', the contents of {\tt db0Guid} can be read.
The function prints the content read from the data block and finally calls ``ocrShutdown''.
