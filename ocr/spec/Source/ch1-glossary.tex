%
% This is an included file. See the master file for more information.
%
%
\section{Glossary}
\label{sec:Glossary}
\glossaryterm{Acquired}
\index{Acquired}
\glossarydefstart
The state of a data block when its chunk of data is accessible to an
OCR object. For example, an EDT must acquire a data block before it
can read-from or write-to that data block.
\glossarydefend

\glossaryterm{Data Block (DB)}
\index{Data Block}\index{DB}
\glossarydefstart
The data, used by an OCR object such as an EDT, that is intended for
access by other OCR objects. A data block specifies
a chunk of data that is entirely accessible as an offset from a starting address.
\glossarydefend

\glossaryterm{Dependence}
\index{Dependence}
\glossarydefstart
A dependence is a link between the post-slot of a source event,
data block or EDT and the pre-slot of a destination EDT or event. The
satisfaction of the source OCR object's post-slot will trigger the
satisfaction of the destination OCR object's pre-slot.
\glossarydefend

\glossaryterm{Event Driven Task (EDT)}
\index{Event Driven Task}\index{EDT}
\glossarydefstart
An OCR object that implements the concept of a task. An EDT with $N$
dependences will have $N$ \emph{pre-slots} numbered from $0$ to $N-1$
and one post-slot.
Each of the \emph{pre-slots} associated with an EDT connects to a
single OCR object, while the EDT’s single \emph{post-slot} can connect
to multiple OCR objects. An EDT transitions to the \emph{runnable} state when all its
pre-slots have been satisfied; the pre-slots determine which
data blocks, if any, the EDT may access. In other words, an EDT can
only read and write to data blocks that are passed along one of its
\emph{pre-slot} or that it creates. Once an EDT is runnable, it is guaranteed to
eventually run unless the program terminates early.
\glossarydefend
% TODO: Would a figure of EDT pre- and post-slots add clarity here?

\glossaryterm{EDT function}
\index{EDT function}
\glossarydefstart
The function that defines the code to be executed by an EDT. An EDT function
takes as arguments the number of parameters, the actual array of
parameters, the number of dependences and the actual array of
dependences. \emph{Parameters} are static 64-bit values known at EDT
creation time. \emph{Dependences} are dynamic control or data
dependences and are also refered to as the EDT's pre-slots. The
parameter array is copied by value when the EDT is created and enters
the \emph{available} state. The dependences (namely the array of dependences) are
determined at runtime and are fully resolved when the EDT enters the
\emph{resolved} state. An EDT function's return type is a GUID. The GUID returned
will be passed along to the EDT's post-slot.
\glossarydefend

\glossaryterm{EDT template}
\index{EDT template}
\glossarydefstart
An OCR object from which an EDT instance is created. The EDT template
stores meta-data related to the EDT definition: the EDT function, and
the number of parameters and dependences available to EDTs
instantiated (created) from this template. Multiple EDTs can be created from the
same EDT template.
\glossarydefend

% TODO: Organize the event definitions along the lines of OpenMP spec: high-level
% topical breakdown & alphabetical inside each topic
\glossaryterm{Event}
\index{Event}
\glossarydefstart
An OCR object used as an indirection mechanism between other OCR
objects interested in each other's change of state (unsatisfied to
satisfied). Events are the main synchronization mechanism in OCR.
\glossarydefend

\glossaryterm{Finish EDT}
\index{Finish EDT}
\glossarydefstart
A special class of EDT. As an EDT runs, it may create additional EDTs
which may themselves create even more EDTs.
In the case of a finish EDT, the EDT's post-slot will only be
satisfied after the EDTs created within its scope (i.e.\ its child EDTs and further
descendants) complete and satisfy their post-slots or are destroyed prior to
becoming runnable. The result is that
any OCR object linked to the post-slot of the finish EDT will, by
necessity, not become runnable (i.e. be scheduled for
execution) until the finish EDT and all EDTs created during its
execution have either completed or been destroyed by the user.
\glossarydefend

\glossaryterm{Globally Unique ID (GUID)}
\index{Globally Unique ID}\index{GUID}
\glossarydefstart
A value generated by the runtime system that uniquely identifies each
OCR object. The GUIDs for the OCR objects reside in a global namespace
visible to all EDTs. Creation and management
of GUIDs is managed by the OCR runtime by default, though user input can also be
used to influence GUID allocation, if so desired.
\glossarydefend

\glossaryterm{Latch Event}
\index{Latch Event}
\glossarydefstart
A special type of event that propagates a satisfy signal to its post-slot
when it has been satisfied an equal number of times on each of its two
pre-slots. In other words, if you imagine a monotonically increasing
counter initially set to zero, on each of the two pre-slots, the latch event's post-slot will
be satisfied if and only if both monotonically
increasing counters are non-zero and equal. Note that once the latch
event's post-slot is satisfied, satisfaction on the latch event's
pre-slots will result in undefined behavior; the latch event will
therefore only satisfy its post-slot at most one time.
\glossarydefend

% TODO: consider an OO-style abstraction for events, in which an
%event has a void satisfy function with a parameter list that includes the pointer to
%the satisfy function, and an argc, argc pair that represents arguments for the function.
% The internal state of the event is stored in its metadata.

\glossaryterm{Link}
\index{Link}
\glossarydefstart
A dependence between OCR objects typically expressed as a connection
between the post-slot of one OCR object and a pre-slot of
another. For example if there is a path between the post-slot of a data block and
the pre-slot of an EDT, the data block is said to be ``linked to the EDT''.
\glossarydefend

\glossaryterm{OCR object}
\index{OCR object}
\glossarydefstart
An object managed by OCR. \emph{EDTs}, \emph{events},
\emph{EDT templates}, and \emph{data blocks} are the most frequently
encountered examples of OCR objects. Each OCR object has a unique
identifier, or GUID.
\glossarydefend

\glossaryterm{OCR program}
\index{OCR program}
\glossarydefstart
A program that is conformant to the OCR specification. Statements in
the OCR specification about the OCR program only refer to behaviors
associated with the constructs that make up OCR. For example, if an
OCR program were to use a parallel programming model outside of OCR,
that program is no longer a purely conformant OCR program and its
behavior could no longer be defined by OCR.
\glossarydefend

\glossaryterm{Released}
\index{Released}
\glossarydefstart
The state of a linked data block that is no longer accessible by a certain OCR
object. For example, after an EDT has finished all of its
modification to a data block and it is ready to make those
modifications accessible by other EDTs, it must release that data
block.
\glossarydefend

\glossaryterm{Slot}
\index{Slot}
\index{Pre-slot}
\index{Post-slot}
\glossarydefstart
Positional end point for a dependence. An OCR object has one or more
slots. Exactly one slot is a \emph{post-slot}. This is used to
communicate the state of the OCR object to other OCR objects. The
other zero or more slots are \emph{pre-slots}, which are used to manage
input dependences of the OCR object. A slot can be:
\begin{itemize}
\item Unconnected: there are no links connecting to the slot;
\item Connected: a link attaches a source's post-slot to a destination pre-slot.
\end{itemize}
A slot in the \emph{connected} state can be:
\begin{itemize}
\item Satisfied: the source of the link has been triggered;
%\item Triggered: see 'Trigger';
\item Unsatisfied: the source of the link has not been triggered.
\end{itemize}
\glossarydefend

\glossaryterm{Task}
\index{Task}
\glossarydefstart
A non-blocking set of instructions that constitute the fundamental
``unit of work'' in OCR. By ``non-blocking'' we mean that once all
preconditions on a task are met, the task is runnable
and it will execute at some point, regardless of what any other
task in the system does. The concept of a task is realized by
the OCR object ``Event Driven Task'' or EDT.
\glossarydefend

\glossaryterm{Trigger}
\index{Trigger}
\glossarydefstart
This term is used to describe the action of either a ``satisfied''
post-slot or of an event whose trigger rule is satisfied. In the
former case, when a post-slot on
an OCR object is satisfied, it triggers any connected
pre-slots causing them to become ``satisfied''. In the latter case,
when an event's trigger rule is satisfied (due to satisfaction(s) on its
pre-slot(s)), it satisfies its post-slot. Therefore, for most events,
when the event's pre-slot becomes satisfied, this will {\bf a)} trigger
the event causing it to satisfy its post-slot and {\bf b)} trigger
the dependence link and satisfy all pre-slots connected to the event's
post-slot. The conjugated form \emph{triggered} is used as an
attributive past participle; that is:
``an EDT that has finished executing the code in its EDT function and
released its data blocks will satisfy the event associated
with its post-slot and become a triggered EDT''.
\glossarydefend


\glossaryterm{Unit of Execution}
\index{Unit of Execution}\index{UE}
\glossarydefstart
A generic term for a process, thread, or any other executable agent
that carries out the work associated with a program.
\glossarydefend

% TODO: consider and more carefully define the idea of the work associated with a
% computation. More details on Vivek's feedback to v1.0 spec

\glossaryterm{Worker}
\index{Worker}
\glossarydefstart
The unit of execution (e.g.\ a process or a thread) that carries out
the sequence of instructions associated with the EDTs in an OCR
program. The details of a worker are tied to a particular
implementation of an OCR platform and are not defined by OCR.
\glossarydefend

% This is the end of ch1-glossary of the OCR specification.
