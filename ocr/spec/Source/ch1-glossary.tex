%
% This is an included file. See the master file for more information.
%

%
\section{Glossary}
\label{sec:Glossary}
\glossaryterm{Acquired}
\index{Acquired}
\glossarydefstart
The state of a data block when its chunk of data is accessible to an
OCR object. For example, an EDT must acquire a data block before it
can read-from or write-to that data block.
\glossarydefend

\glossaryterm{Data Block (DB)}
\index{Data Block}
\glossarydefstart
The data, used by an OCR object such as an EDT, that is intended for
access by other OCR objects. A data block specifies
a chunk of data that is entirely accessible as an offset from a starting address.
\glossarydefend

\glossaryterm{Dependence}
\index{Dependence}
\glossarydefstart
A dependence is a link between the post-slot of a source event or
data-block and the pre-slot of a destination EDT or event. The
satisfaction of the source OCR object's post-slot will trigger the
satisfaction of the destination OCR object's pre-slot.
\glossarydefend

\glossaryterm{Event driven Task (EDT)}
\index{Event Driven Task}\index{EDT}
\glossarydefstart
An OCR object that implements the concept of a task. An EDT with $N$
dependences will have $N$ \emph{pre-slots} numbered from $0$ to $N-1$
and one post-slot.
Each of the \emph{pre-slots} associated with an EDT connects to a
single OCR object, while the EDT’s single \emph{post-slot} can connect
to multiple OCR objects. An EDT transitions to the \emph{ready} state when all its
pre-slots have been satisfied; the pre-slots determine which
data blocks, if any, the EDT may access.  Once an EDT is in the ready state, it will
eventually run on the OCR platform.
\glossarydefend

% Vivek writes ...
%An EDT may have zero or more pre-slots that serve as preconditions. Each pre-slot is associated with a
% single OCR event. An EDT also has a single post-slot that is associated with the event that represents
%the completion of the EDT's execution.
%
%NOTE: I really like the text below on how the DAG is defined, but I'm not sure that it belongs in
%the glossary. It might be better to create a separate section titled something like "Computation DAG"
%and use that to explicitly define the DAG referred to in Section 1.1. That new subsection should include
%figures, with a graphical notation (e.g.,with input ports for each of the pre-slots) that makes the points
% in the text very clear. I'd be happy to provide text for that  subsection, if needed.
%
% TGM responds ... I like much of what Vivek is saying here, but we don't have time to update the
%definition as he suggested (we'd need to define a continuation for example). We may choose to
% add that section, but we'll have to also discuss that for a later day. The problem is that every
% presentation I've seen that uses the graphical representation vivek alludes to makes things
% less clear, not more clear. This is the same problem I have with CnC which is only clear
% and easy to understand by the handful or people who created it or worked directly with it. To
% an application programmer, their diagrams are just a confusing mess. The same holds for OCR
% so we need to figure out how to fix this before we work it into the spec.
\glossaryterm{EDT function}
\index{EDT function}
\glossarydefstart
The function that defines the code to be executed by an EDT. The function
takes as arguments the number of parameters, the actual array of
parameters, the number of dependences and the actual array of
dependences. \emph{Parameters} are static 64-bit values known at EDT
creation time and \emph{dependences} are dynamic control or data
dependences. The parameter array is copied by value when the EDT is
created and enters the \emph{available} state. The dependences (namely the array of dependences) are
determined at runtime and are fully resolved only when the EDT is launched and
is ready to execute. The EDT function can optionally return the GUID of a Data Block or
event that will be passed along its ``post'' slot.

\glossarydefend
\glossaryterm{EDT template}
\index{EDT template}
\glossarydefstart
An OCR object from which an EDT instance is created. The EDT template
stores meta-data related to the EDT definition, the EDT function, and
the number of parameters and dependences available to EDTs
instantiated (created) from this template. Multiple EDTs can be created from the
same EDT template.
\glossarydefend


%%
%TGM:  Rob urges us to put all the event definitions together. We may want to organize this glossary along the lines
% of what is done in the OpenMP specification .... a high level topical break-down and alphabetical inside
% at topic.

\glossaryterm{Event}
\index{Event}
\glossarydefstart
An OCR object used as an indirection mechanism between other OCR
objects interested in each other's change of state (unsatisfied to
satisfied). Events are the main synchronization mechanism in OCR.
\glossarydefend

\glossaryterm{Finish EDT}
\index{Finish EDT}
\glossarydefstart
A special class of EDT. As an EDT runs, it may create additional EDTs
which may themselves create even more EDTs.
For the case of a finish
EDT, the EDTs created within its scope (i.e.\ its child EDTs and further
descendants) complete and satisfy their post-slots
before the finish EDT can satisfy its post-slot. The result is that
any OCR object linked to the post-slot of the finish EDT will by
necessity not be enter the \emph{ready} state (i.e. be scheduled for
execution) until the finish EDT and all
EDTs created during its execution have completed.
%
% The following text from Vivek is more precise, but it generates a latex error when I
% try to typeset it.
%
%For the case of a Finish EDT, F, each EDT created with F identified
%as its Immediately Enclosing Finish (IEF)
%must complete execution and satisfy its post-slot before Finish EDT F can satisfy its post-slot.
%The result is that any OCR object linked to the post-slot of
%the finish~EDT will by necessity
%wait until finish~EDT F and all EDTs that have F as their IEF have completed.

\glossarydefend


\glossaryterm{Globally Unique ID (GUID)}
\index{Globally Unique ID}\index{GUID}
\glossarydefstart
A value generated by the runtime system that uniquely identifies each
OCR object. The GUIDs for the OCR objects reside in a global name
space visible to all EDTs.
\glossarydefend

% Vivek writes ...
%
%  A 64-bit value that uniquely identifies each OCR object. The GUIDs for the OCR
% objects reside in a global name space visible to all EDTs. Creation and management
% of GUIDs is managed by the OCR runtime by default, though user input can also be
% used to influence GUID allocation, if so desired.
%
% TGM responds ... two problem. Why does it have to be 64 bit as defined in the spec?   I don't
% get that. Second, the current API doesn't provide user input to influence GUID allocation so
% we can't say this yet. Right?

\glossaryterm{Latch Event}
\index{Latch Event}
\glossarydefstart
A special type of event that propagates a satisfy signal to its post-slot
when it has been satisfied an equal number of times on each of its two
pre-slots. In other words, if you imagine a
monotonically increasing counter on each of the two pre-slots, the
latch event's post-slot will be satisfied if and only if both monotonically
increasing counters are non-zero and equal. Note that once the latch
event's post-slot is satisfied, satisfaction on the latch event's
pre-slots will result in undefined behavior; the latch event will
therefore only satisfy its post-slot at most one time.
\glossarydefend

% vivek writes ...
%(See earlier note re. definition of pre-slots and post-slots for events.)
%
%Also, a problematic aspect of a latch event is that it appears that a pre-slot can
%be "triggered" multiple times. Maybe it should not be called a pre-slot?
%
%In general, it feels like we need an OO-style abstraction for events, in which an
%event has a void satisfy function with a parameter list that includes the pointer to
%the satisfy function, and an argc, argc pair that represents arguments for the function.
% One challenge is where to store the internal state of the event. Seems like it needs
% to be in a data block?

%TGM responds ... this is another case where I didnt' want to make major changes to
% this without a long discussion with the group. I can't tell you how much time we  spent
%creating the current definition so changes would be problematic.
%
% I wanted to capture Vivek's comments espeically about the OO suggestion. I think he
% is touching on something very important there and we need to dsicuss this in more
% detail.
%
% REC: I agree with his OO-style of abstraction. I have always (but
% probably not very clearly) tried to say that events have a mini
% ``rule'' that determines when its post-slot get satisfied. As for
% the state to store, it's in the metadata


\glossaryterm{Link}
\index{Link}
\glossarydefstart
A dependence between OCR objects typically expressed as a connection
between the post=slot of one OCR object and the pre-slot of another.  Data-Blocks
exposed through the pre-slots of an EDT are said to be ``linked to the EDT''.
\glossarydefend


\glossaryterm{OCR object}
\index{OCR object}
\glossarydefstart
A reference counted object managed by OCR. \emph{EDTs}, \emph{events},
\emph{templates}, and \emph{data blocks} are the most frequently
encountered examples of OCR objects. Each OCR object has a unique
identifier, or GUID.
\glossarydefend

\glossaryterm{OCR program}
\index{OCR program}
\glossarydefstart
A program that is conformant to the OCR specification. Statements in
the OCR specification about the OCR program only refer to behaviors
associated with the constructs that make up OCR. For example, if an
OCR program were to use a parallel programming model outside of OCR,
that program is no longer a purely conformant OCR program and its
behavior can no longer be defined by OCR.
\glossarydefend

\glossaryterm{Released}
\index{Released}
\glossarydefstart
The state of a data block that is no longer accessible by a certain OCR
object. For example, after an EDT has finished all of its
modification to a data block and it is ready to make those
modifications accessible by other EDTs, it must release that data
block.
\glossarydefend

\glossaryterm{Slot}
\index{Slot}
\glossarydefstart
Positional end point for a dependence. An OCR object has one or more
slots. Exactly one slot is a \emph{post-slot}. This is used to
communicate the state of the OCR object to other OCR objects. The
other zero of more slots are \emph{pre-slots}, which are used to manage
input dependences of the OCR object. A slot can be:
\begin{itemize}
\item Unconnected: There are no links connecting to the slot;
\item Connected: a link attaches a source post-slot to a destination pre-slot.
\end{itemize}
A slot in the \emph{connected} state can be:
\begin{itemize}
\item Satisfied: the source of the link has been triggered;
%\item Triggered: see 'Trigger';
\item Unsatisfied: the source of the link has not been triggered.
\end{itemize}
\glossarydefend

\glossaryterm{Task}
\index{Task}
\glossarydefstart
A non-blocking set of instructions that constitute the fundamental
``unit of work'' in OCR.  By ``non-blocking'' we mean that once all
preconditions on a task are met, the task is ``ready to execute''
and it will execute at some point,
regardless of what any other task in the system does. The concept of
a task is realized by the OCR object ``Event Driven Task'' or EDT.
\glossarydefend

\glossaryterm{Trigger}
\index{Trigger}
\glossarydefstart
This term is used to describe the action of either a ``satisfied''
post-slot or of an event whose trigger rule is satisfied. In the
former case, when a post slot on
an OCR object is satisfied, it triggers any connected
pre-slots causing them to become ``satisfied''. In the latter case,
when an event's trigger rule is satisfied (due to satisfaction(s) on its
pre-slot(s)), it satisfies its post-slot. Therefore, for most events,
when the event's pre-slot becomes satisfied, this will trigger the event and
therefore cause it to satisfy its post-slot which will in turn trigger
the dependence link and satisfy all pre-slots connected to the event's
post-slot. The conjugated form \emph{triggered} is used as an
attributive past participle; that is
``an EDT that has finished executing the code in its EDT function and
released its data blocks will satisfy the event associated
with its post-slot and become a triggered EDT''.
\glossarydefend


\glossaryterm{Unit of Execution}
\index{Unit of Execution}\index{UE}
\glossarydefstart
A generic term for a process, thread, or any other executable agent
that carries out the work associated with a program.
\glossarydefend

% vivek wants us to consider and more carefully define the idea of the work associated with a
% computation. See his feedback on the version 1.0 spec for details.


\glossaryterm{Worker}
\index{Worker}
\glossarydefstart
The unit of execution (e.g.\ a process or a thread) that carries out
the sequence of instructions associated with the EDTs in an OCR
program. The details of a worker are tied to a particular
implementation of an OCR platform and are not defined by OCR.
\glossarydefend

% This is the end of ch1-glossary of the OCR specification.