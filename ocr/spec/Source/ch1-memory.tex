%
% This is an included file. See the master file for more information.
%

%vivek writes ...
%
%GENERAL COMMENTS:
%
%
%1) A key challenge in defining a memory model is defining the semantics of a data race. All
% memory models usually have the same semantics for data-race-free programs, but vary vastly
% in their definitions of the semantics of data races. For example, I believe that the original
%Release Consistency model had the "memory coherence" assumption which stated that all
% processors see writes to the same location in the same order, which is likely too strong a
%condition for OCR. For a weaker memory model, see Location Consistency (Gao & Sarkar, IEEE ToC 2000).
%

\section{Memory Model}
\index{Memory Model}
\label{sec:MemoryModel}

A memory model defines the values that can be legally observed in
memory when multiple units of execution (e.g.\ processes or threads)
access a shared memory system. The memory model provides programmers
with the tools they need to understand the state of memory, but it
also places restrictions on what a compiler writer can do (e.g.\ which
aggressive optimizations are allowed) and restrictions on what a
hardware designer is allowed to do (e.g.\ the behavior of write
buffers).

To construct a memory model for OCR, we need to present a few
definitions. The operations inside a task execute in a non-blocking
manner. The order of such operations are defined by the
\emph{sequenced-before}\index{sequenced-before} relations defined by
the host C programming language.

When multiple EDTs are running, they execute asynchronously. Usually,
a programmer can make few assumptions about the relative orders of
operations in two different EDTs. At certain points in the execution
of EDTs, however, the OCR program may need to define ordering
constraints. These constraints define
\emph{synchronized-with}\index{synchronized-with} relations.

The ``transitive closure'' of sequenced-before operations inside each
of two EDTs combined with the synchronized-with relations between two
EDTs defines a \emph{happens-before}\index{happens-before}
relationship. For example:
\begin{itemize}
\item if \code{A} is sequenced-before \code{B} in EDT1
\item  if \code{C} is sequenced-before \code{D} in EDT2
\item and  \code{B} is synchronized-with  \code{C} in EDT2
\item then \code{A} happens-before \code{D}.
\end{itemize}
These basic concepts are enough to define the memory model for OCR.


OCR provides a relatively simple memory model. Before an EDT can read
or write a data block, it must first \emph{acquire}\index{Acquire} the data
block. This is not an exclusive relationship by which we mean it is
possible (depending on the mode of the data block in question) for
multiple EDTs to acquire the same data block at the same
time.  When an EDT has finished with a data block and it is ready
to expose any modifications to the data block to other EDTs, it
must \emph{release}\index{Release} that data block.

\begin{quote}
Any function in the OCR runtime that releases a data block must assure
that all loads and stores to the data block occur before the data
block is released and that the release must complete before the
function returns.
\end{quote}

The only way to establish a synchronized-with relation is through the
behavior of events. If the pre-slot of EDT2 is connected to the post-slot
of EDT1, then EDT2 waits for event associated with the post-slot of EDT1 to
trigger. Therefore, the satisfy event from EDT1 synchronizes-with the
triggering of the pre-slot of EDT2. We can establish a happens-before
relationship between the two EDTs if we define the following rule for
OCR.

\begin{quote}
An EDT must complete the release of all of its resources before it
marks its post-event as satisfied.
\end{quote}

An EDT can use data blocks to satisfy events in the body of the task
in addition to the event associated with its post-slot. We can reason
about the behavior of the memory model and establish happens-before
relationship if we define the following rule.

%Vincent's feedback....
%
%"If an EDT uses a Data Block to satisfy an event, all writes to that data block from the EDT must complete
%before the event is triggered.” => I think we already discussed that but I believe you should wait for all the
%db releases that are sequenced-before the satisfy.
%1) This is in case of a db storing guids. db1 has data the edt has written to,  db2 has the guid of db1. %Satisfying ev1 with db2, potentially enables edt2, which spawns edt3 that depends on db1’s guid read from
%db2. With the current rule, db1 may or may not have been released when edt3 acquires it.
%
%db1[0]=1
%db2[0]=db1Guid
%ocrDbRelease(db1)
%ocrDbRelease(db2)
%ocrEventSatisfy(ev1,db2)
%
% TGM responds:
% I submit that since ocrDbRelease(db1) is sequenced before ocrDbRelease(db2) the rules in the memory
%model forces the releases to occur in that order and therefore the model already covers your case.     Can
%you tell me what I'm missing ... because I agree with you completely that the model must force the releases
%to occur in that order prior to the satisfy ... I just think the normal sequenced before relations cover this.
%




\begin{quote}
If an EDT uses a Data Block to satisfy an event, all writes to that data block
from the EDT must complete before the event is triggered.
\end{quote}

Without this rule we can not assume a release operation followed by
satisfying an event defines a sequenced-before relationship that can
be used to establish a happens-before relation.

The core idea in the OCR memory model is that happens-before
relationships are defined in terms of events (the only synchronization
operation in OCR) and the release of OCR objects (such as data
blocks). This is an instance of a \emph{Release
Consistency}\index{Release Consistency} memory model which has the
advantage of being relatively straightforward to apply to OCR
programs.

The safest course for a programmer is to write programs that can be
defined strictly in terms of the release consistency rules. OCR,
however, lets a programmer write programs in which two or more EDTs can
write to a single data block at the same time (or more precisely, the
two EDTs can issue writes in an unordered manner). This may result in a
data race\index{data race} in that the value that is ultimately stored
in memory depends on how a system chooses to schedule operations from
the two EDTs.

Most modern parallel programming languages state that a program that
has a data race\footnote{A \emph{data race} occurs when loads and
stores by two units of execution operate on overlapping memory regions
without a synchronized-with relation to order them} is an illegal
program and the results produced by such a program are undefined.
These programming models then define a complex set of synchronization
constructs and atomic variables so a programmer has the tools needed
to write race-free programs. OCR, however, does not provide any
synchronization constructs beyond the behavior of events. This is not
an oversight. Rather, this restricted synchronization model helps OCR
to better scale on a wider range of parallel computers.

OCR, therefore, allows a programmer to write legal programs that may
potentially contain data races. OCR deals with this situation by
adding two more rules. In both of these rules, we say that address
range $A$ and $B$ are non-overlapping if and only if the set $A_1$ of
8-byte\footnote{The reference to ``8-byte'' words assumes the processing elements
utilize a 64-bit architecture.  For other cases, all references to an 8-byte
word in this specification must be adjusted to match the architecture of the processing elements.}
aligned 8-byte words fully covering $A$ and the set $B_1$ of
8-byte aligned 8-byte words fully covering $B$ do not overlap. For
example, addresses
$0x0$ and $0x7$ overlap (assuming byte level addressing) whereas
$0x0$ and $0x8$ do not.
The first rule deals with the situation of multiple
EDTs writing to a data block with non-overlapping address ranges.
\begin{quote}
If two EDTs write to a single data block without a well defined order,
if the address ranges of the writes do not overlap, the correct result
of each write operation must appear in memory.
\end{quote}

This behavior may seem obvious making it trivial for a system to
support. However, when addresses are distinct but happen to share the
same cache lines or when aggressive optimization of writes occur
through write buffers, an implementation could mask the updates from
one of the EDTs if this rule were not defined in the OCR
specification.

The last rule addresses the case of overlapping address ranges.  Assume that
a system writes values to memory at an atomicity of N-bytes.  This defines
the fundamental store-atomicity for the system\index{store-atomicity}.
\begin{quote}
If two EDTs write to a single data block without a well defined order,
if the address ranges of the writes overlap, the results written to
memory must correspond to one of the legal interleavings of statements
from the two EDTs at an N-byte aligned granularity. Overlapping writes
to non-aligned or smaller than N-byte granularity are not defined.
\end{quote}
For systems that do not provide store-atomicity at any level, N would be 0
and the above rule states that unordered writes to overlapping address
ranges are undefined.
This rule is the well known \emph{sequential consistency}\index{sequential
consistency} rule. It states that the actual result appearing in
memory may be nondeterministic, but it will be well defined and it
will correspond to values from one EDT or the other.

Release consistency remains the safest and best approach to use in
writing OCR programs. It is conceivable that some of the more
difficult rules may be relaxed in future versions of OCR (especially
the sequential consistency rule), but the relaxed consistency model
will almost assuredly always be supported by OCR.

Any memory in OCR that can be accessed by multiple EDTs resides in data blocks.  As
discussed in section~\ref{sec:datablocks} there
are four access modes for the data blocks in OCR. The modes and how they
interact with the OCR memory model are listed as follows.
\begin{itemize}

\item \emph{Read-Write} (default mode)\index{Data Block, read-write}: The EDT may read
and write to the data block.  Multiple EDTs may
write to the same data block at the same time with values constrained according to the
rules in the OCR memory model.

\item \emph{Exclusive write}\index{Data Block, exclusive write}: The
EDT requires that it is the only EDT that can commit write operations to a data block at a
given time.  Writes must follow a sequential total order; i.e.\ when more than one EDT is writing
to a data block in exclusive write mode, all the writes from one EDT must complete before
a subsequent EDT can acquire and then write to the data block.  This serializes the execution of
EDTs that acquire data blocks in exclusive write mode.

\item \emph{Read only}\index{Data Block, read only}: The EDT
will only read from the Data Block. The OCR Runtime does
not restrict the ability of other EDTs to write to the data block.  The
visibility of those writes are undefined; i.e.\ an implementation may
choose whether or not to make writes by other EDTs visible.

\item \emph{Constant}\index{Data Block,constant}: The EDT will only read from the
data block and the OCR runtime will assure that once the data block is acquired,
writes from other EDTs will not be visible.

\end{itemize}


% This is the end of ch1-memory.tex of the OCR specification.