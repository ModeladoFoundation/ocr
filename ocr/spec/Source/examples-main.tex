This chapter demonstrates the use of OCR through a series of
examples. The examples are ordered from the most basic to the most
complicated and frequently make use of previous examples. They are
meant to guide the reader in understanding the fundamental concepts of
the OCR programming model and API.

% Ideas for putting examples here:
%   - We should figure out what APIs or model characteristics we want
%   to demonstrate in each example (and put it here so we know)
%   - Each code snippet should show that feature and allow for the
%   entire example to be constructed from that snippet and any
%   previous snippets (ie: we don't have to put the whole program
%   everytime but it should be ``obvious'' how to build it)

%%%%%
\section{OCR's ``Hello World!''}
% In this example:
%   - APIs: ocrMain, ocrShutdown, PRINTF, getArgc, getArgv
%   - Functionality: program entry point, argument passing, shutdown
%   - Concepts: None
This example illustrates the most basic OCR program: a single function
that prints the message ``Hello World!'' on the screen and exits.
%%%%
\subsection{Code example}
The following code will print the string ``Hello World!'' to the
standard output and exit. Note that this program is fully functional
(ie: there is no need for a \texttt{main} function).

\begin{ocrsnip}
#include <ocr.h> (@ \label{line:HW_include} @)

ocrGuid_t mainEdt(u32 paramc, u64* paramv, u32 depc, ocrEdtDep_t depv[]) { (@ \label{line:HW_mainEdt} @)
    PRINTF('Hello World!\n'); (@ \label{line:HW_printf}@)
    ocrShutdown(); (@ \label{line:HW_shutdown}@)
    return NULL_GUID;
}
\end{ocrsnip}
%%%
\subsubsection{Details}
The \texttt{ocr.h} file included on Line~\ref{line:HW_include}
contains all of the main OCR APIs. Other more experimental or extended
APIs are also located in the \texttt{extensions/} folder of the
include directory.

EDT's signature is shown on Line~\ref{line:HW_mainEdt}. A special EDT,
named \texttt{mainEdt} is called by the runtime if the programmer does not
provide a \texttt{main} function\footnote{Note that if the programmer
  \emph{does} provide a \texttt{main} function, it is the
  responsability of the programmer to properly initialize the runtime,
  call the first EDT to execute and properly shutdown the
  runtime. Refer to the legacy mode extension and the
  \texttt{ocr-legacy.h} header file for more detail.}.

The \texttt{ocrShutdown} function called on
Line~\ref{line:HW_shutdown} should be called once and only once by all
OCR programs to indicate that the program has terminated. The runtime
will then shutdown and any non-executed EDTs at that time are not
guaranteed to execute.

%%%%%
\section{Expressing a Fork-Join pattern}
This example illustrates the creation of a fork-join pattern in OCR.

%%%%
\subsection{Code example}
\begin{ocrsnip}
/* Example of a "fork-join" pattern in OCR
 *
 * Implements the following dependence graph:
 *
 *   mainEdt
 *   /    \
 * fun1   fun2
 *   \    /
 * shutdownEdt
 *
 */

#include "ocr.h"  (@ \label{line:FJ_include} @)

ocrGuid_t fun1(u32 paramc, u64* paramv, u32 depc, ocrEdtDep_t depv[]) {
    int* k;
    ocrGuid_t db_guid;
    ocrDbCreate(&db_guid,(void **) &k, sizeof(int), 0, NULL_HINT, NO_ALLOC); (@ \label{line:FJ_db1}@)
    k[0]=1; (@ \label{line:FJ_k1}@)
    PRINTF("Hello from fun1, sending k = %lu\n",*k);
    return db_guid; (@ \label{line:FJ_retDB1}@)
}

ocrGuid_t fun2(u32 paramc, u64* paramv, u32 depc, ocrEdtDep_t depv[]) {
    int* k;
    ocrGuid_t db_guid;
    ocrDbCreate(&db_guid,(void **) &k, sizeof(int), 0, NULL_HINT, NO_ALLOC); (@ \label{line:FJ_db2}@)
    k[0]=2; (@ \label{line:FJ_k2}@)
    PRINTF("Hello from fun2, sending k = %lu\n",*k);
    return db_guid; (@ \label{line:FJ_retDB2}@)
}

ocrGuid_t shutdownEdt(u32 paramc, u64* paramv, u32 depc, ocrEdtDep_t depv[]) { (@ \label{line:FJ_shutdown}@)
    PRINTF("Hello from shutdownEdt\n");
    int* data1 = (int*) depv[0].ptr;
    int* data2 = (int*) depv[1].ptr;
    PRINTF("Received data1 = %lu, data2 = %lu\n", *data1, *data2);
    ocrDbDestroy(depv[0].guid);
    ocrDbDestroy(depv[1].guid);
    ocrShutdown();
    return NULL_GUID;
}

ocrGuid_t mainEdt(u32 paramc, u64* paramv, u32 depc, ocrEdtDep_t depv[]) { (@ \label{line:FJ_mainEdt} @)
    PRINTF("Starting mainEdt\n");
    ocrGuid_t edt1_template, edt2_template, edt3_template;
    ocrGuid_t edt1, edt2, edt3, outputEvent1, outputEvent2;

    //Create templates for the EDTs
    ocrEdtTemplateCreate(&edt1_template, fun1, 0, 1); (@ \label{line:FJ_edtTemplt1} @)
    ocrEdtTemplateCreate(&edt2_template, fun2, 0, 1); (@ \label{line:FJ_edtTemplt2} @)
    ocrEdtTemplateCreate(&edt3_template, shutdownEdt, 0, 2);  (@ \label{line:FJ_edtTemplt3} @)

    //Create the EDTs
    ocrEdtCreate(&edt1, edt1_template, EDT_PARAM_DEF, NULL, EDT_PARAM_DEF, NULL, EDT_PROP_NONE, NULL_HINT, &outputEvent1); (@ \label{line:FJ_edt1} @)
    ocrEdtCreate(&edt2, edt2_template, EDT_PARAM_DEF, NULL, EDT_PARAM_DEF, NULL, EDT_PROP_NONE, NULL_HINT, &outputEvent2); (@ \label{line:FJ_edt2} @)
    ocrEdtCreate(&edt3, edt3_template, EDT_PARAM_DEF, NULL, 2, NULL, EDT_PROP_NONE, NULL_HINT, NULL); (@ \label{line:FJ_edt3} @)

    //Setup dependences for the shutdown EDT
    ocrAddDependence(outputEvent1, edt3, 0, DB_MODE_CONST); (@ \label{line:FJ_edt3Dep1} @)
    ocrAddDependence(outputEvent2, edt3, 1, DB_MODE_CONST); (@ \label{line:FJ_edt3Dep2} @)

    //Start execution of the parallel EDTs
    ocrAddDependence(NULL_GUID, edt1, 0, DB_DEFAULT_MODE); (@ \label{line:FJ_edt1start} @)
    ocrAddDependence(NULL_GUID, edt2, 0, DB_DEFAULT_MODE); (@ \label{line:FJ_edt2start} @)
    return NULL_GUID;
}
\end{ocrsnip}
%%%
\subsubsection{Details}

%%%%
The \texttt{ocr.h} file included on Line~\ref{line:FJ_include}
contains all of the main OCR APIs. The \texttt{mainEdt} is
shown on Line~\ref{line:FJ_mainEdt}. It is called by the runtime
as a \texttt{main} function is not provided
(more details in \texttt{hello.c}).

The \texttt{mainEdt} creates three templates (Lines~\ref{line:FJ_edtTemplt1}, ~\ref{line:FJ_edtTemplt2}
and ~\ref{line:FJ_edtTemplt3}), respectively for three different EDTs
(Lines~\ref{line:FJ_edt1}, ~\ref{line:FJ_edt2} and ~\ref{line:FJ_edt3}).
An EDT is created as an instance of an EDT template.
This template stores metadata about EDT,
optionally defines the number of dependences and parameters used when
creating an instance of an EDT, and is a container for the function
that will be executed by an EDT. This function is called the EDT function.
For the EDTs, \texttt{edt1}, \texttt{edt2} and \texttt{edt3}, the EDT functions
are, \texttt{fun1}, \texttt{fun2} and \texttt{shutdownEdt}, respectively.
The last parameter to \texttt{ocrEdtTemplateCreate} is the total number of
data blocks on which the EDTs depends. The signature of EDT creation API,
\texttt{ocrEdtCreate}, is shown in Lines~\ref{line:FJ_edt1}, ~\ref{line:FJ_edt2}
and ~\ref{line:FJ_edt3}. When \texttt{edt1} and \texttt{edt2} will complete,
they will satisfy the output events \texttt{outputEvent1} and \texttt{outputEvent2} repectively.
This is not required for \texttt{edt3}. However, \texttt{edt3} should execute only
when the events \texttt{outputEvent1} and \texttt{outputEvent2} are satisfied.
This is done by setting up dependencies on \texttt{edt3} by using the API \texttt{ocrAddDependence},
as shown in Lines~\ref{line:FJ_edt3Dep1} and ~\ref{line:FJ_edt3Dep2}.
This spawns \texttt{edt3} but it will not execute until both the events
are satisfied. Finally, the EDTs \texttt{edt1} and \texttt{edt2} are
spawned in Lines~\ref{line:FJ_edt1start} and ~\ref{line:FJ_edt2start} respectively.
As they do not have any dependencies, they execute the associated EDT functions
in parallel. These functions (\texttt{fun1} and \texttt{fun2}) creates
data-blocks using the API \texttt{ocrDbCreate} (Lines~\ref{line:FJ_db1} and ~\ref{line:FJ_db2}).
The data is written to the data-blocks and the GUID is returned (Lines~\ref{line:FJ_retDB1}
and ~\ref{line:FJ_retDB2}). This will satisfy the events on which the \texttt{edt3}
is waiting. The EDT function \texttt{shutdownEdt} executes
and calls \texttt{ocrShutdown} after reading and destroying the two data-blocks.

%%%%

%%%%%
\section{Expressing unstructured parallelism}
%%%%
\subsection{Code example}
This example illustrates several aspect of the OCR API with regards to
the creation of an irregular task graph. Specifically, it illustrates:
\begin{enumerate}
\item{Adding dependences between {\bf a)} events and EDTs, {\bf b)}
    data-blocks and EDTs, and {\bf c)} the NULL\_GUID and EDTs;}
\item{The use of an EDT's post-slot and how a ``producer'' EDT can
    pass a data-block to a ``consumer'' EDT using this post-slot;}
\item{Several methods of satisfying an EDT's pre-slot: {\bf a)}
    through the use of an explicit dependence array at creation time,
    {\bf b)} through the use of another EDT's post-slot and {\bf c)}
    through the use of an explicitly added dependence followed by an
    \texttt{ocrEventSatisfy} call.}
\end{enumerate}
\begin{ocrsnip}
/* Example of a pattern that highlights the
 * expressiveness of task dependences
 *
 * Implements the following dependence graph:
 * (@ \label{line:task-dep-graph} @)
 * mainEdt
 * |      \
 * stage1a stage1b
 * |     \       |
 * |      \      |
 * |       \     |
 * stage2a  stage2b
 *     \      /
 *     shutdownEdt
 */
#include "ocr.h"

#define NB_ELEM_DB 20

ocrGuid_t shutdownEdt(u32 paramc, u64* paramv, u32 depc, ocrEdtDep_t depv[]) {
    ASSERT(depc == 2);
    u64* data0 = (u64*)depv[0].ptr;
    u64* data1 = (u64*)depv[1].ptr;

    ASSERT(*data0 == 3ULL);
    ASSERT(*data1 == 4ULL);
    PRINTF("Got a DB (GUID 0x%lx) containing %lu on slot 0\n", depv[0].guid, *data0);
    PRINTF("Got a DB (GUID 0x%lx) containing %lu on slot 1\n", depv[1].guid, *data1);

    // Free the data-blocks that were passed in
    ocrDbDestroy(depv[0].guid);
    ocrDbDestroy(depv[1].guid);

    // Shutdown the runtime
    ocrShutdown();
    return NULL_GUID;
}

ocrGuid_t stage2a(u32 paramc, u64* paramv, u32 depc, ocrEdtDep_t depv[]);

ocrGuid_t stage1a(u32 paramc, u64* paramv, u32 depc, ocrEdtDep_t depv[]) {
    ASSERT(depc == 1);
    ASSERT(paramc == 1);
    // paramv[0] is the event that the child EDT has to satisfy
    // when it is done

    // We create a data-block for one u64 and put data in it
    ocrGuid_t dbGuid = NULL_GUID, stage2aTemplateGuid = NULL_GUID,
        stage2aEdtGuid = NULL_GUID;
    u64* dbPtr = NULL;
    ocrDbCreate(&dbGuid, (void**)&dbPtr, sizeof(u64), 0, NULL_GUID, NO_ALLOC);
    *dbPtr = 1ULL;

    // Create an EDT and pass it the data-block we just created
    // The EDT is immediately ready to execute
    ocrEdtTemplateCreate(&stage2aTemplateGuid, stage2a, 1, 1);
    ocrEdtCreate(&stage2aEdtGuid, stage2aTemplateGuid, EDT_PARAM_DEF,(@ \label{line:expDep} @)
                 paramv, EDT_PARAM_DEF, &dbGuid, EDT_PROP_NONE, NULL_GUID, NULL);

    // Pass the same data-block created to stage2b (links setup in mainEdt)
    return dbGuid; (@ \label{line:EdtGuidReturn} @)
}

ocrGuid_t stage1b(u32 paramc, u64* paramv, u32 depc, ocrEdtDep_t depv[]) {
    ASSERT(depc == 1);
    ASSERT(paramc == 0);

    // We create a data-block for one u64 and put data in it
    ocrGuid_t dbGuid = NULL_GUID;
    u64* dbPtr = NULL;
    ocrDbCreate(&dbGuid, (void**)&dbPtr, sizeof(u64), 0, NULL_GUID, NO_ALLOC);
    *dbPtr = 2ULL;

    // Pass the created data-block created to stage2b (links setup in mainEdt)
    return dbGuid;
}

ocrGuid_t stage2a(u32 paramc, u64* paramv, u32 depc, ocrEdtDep_t depv[]) {
    ASSERT(depc == 1);
    ASSERT(paramc == 1);

    u64 *dbPtr = (u64*)depv[0].ptr;
    ASSERT(*dbPtr == 1ULL); // We got this from stage1a

    *dbPtr = 3ULL; // Update the value

    // Pass the modified data-block to shutdown
    ocrEventSatisfy((ocrGuid_t)paramv[0], depv[0].guid); (@ \label{line:expSatisfy} @)

    return NULL_GUID;
}

ocrGuid_t stage2b(u32 paramc, u64* paramv, u32 depc, ocrEdtDep_t depv[]) {
    ASSERT(depc == 2);
    ASSERT(paramc == 0);

    u64 *dbPtr = (u64*)depv[1].ptr;
    // Here, we can run concurrently to stage2a which modifies the value
    // we see in depv[0].ptr. We should see either 1ULL or 3ULL

    // On depv[1], we get the value from stage1b and it should be 2
    ASSERT(*dbPtr == 2ULL); // We got this from stage2a

    *dbPtr = 4ULL; // Update the value

    return depv[1].guid; // Pass this to the shudown EDT
}


ocrGuid_t mainEdt(u32 paramc, u64* paramv, u32 depc, ocrEdtDep_t depv[]) {

    // Create the shutdown EDT
    ocrGuid_t stage1aTemplateGuid = NULL_GUID, stage1bTemplateGuid = NULL_GUID,
        stage2bTemplateGuid = NULL_GUID, shutdownEdtTemplateGuid = NULL_GUID;
    ocrGuid_t shutdownEdtGuid = NULL_GUID, stage1aEdtGuid = NULL_GUID,
        stage1bEdtGuid = NULL_GUID, stage2bEdtGuid = NULL_GUID,
        evtGuid = NULL_GUID, stage1aOut = NULL_GUID, stage1bOut = NULL_GUID,
        stage2bOut = NULL_GUID;

    ocrEdtTemplateCreate(&shutdownEdtTemplateGuid, shutdownEdt, 0, 2);
    ocrEdtCreate(&shutdownEdtGuid, shutdownEdtTemplateGuid, 0, NULL, EDT_PARAM_DEF, NULL,
                 EDT_PROP_NONE, NULL_GUID, NULL);

    // Create the event to satisfy shutdownEdt by stage 2a
    // (stage 2a is created by 1a)
    ocrEventCreate(&evtGuid, OCR_EVENT_ONCE_T, true);

    // Create stages 1a, 1b and 2b
    // For 1a and 1b, add a "fake" dependence to avoid races between
    // setting up the event links and running the EDT
    ocrEdtTemplateCreate(&stage1aTemplateGuid, stage1a, 1, 1);
    ocrEdtCreate(&stage1aEdtGuid, stage1aTemplateGuid, EDT_PARAM_DEF, &evtGuid,
                 EDT_PARAM_DEF, NULL, EDT_PROP_NONE, NULL_GUID, &stage1aOut);

    ocrEdtTemplateCreate(&stage1bTemplateGuid, stage1b, 0, 1);
    ocrEdtCreate(&stage1bEdtGuid, stage1bTemplateGuid, EDT_PARAM_DEF, NULL,
                 EDT_PARAM_DEF, NULL, EDT_PROP_NONE, NULL_GUID, &stage1bOut);

    ocrEdtTemplateCreate(&stage2bTemplateGuid, stage2b, 0, 2);
    ocrEdtCreate(&stage2bEdtGuid, stage2bTemplateGuid, EDT_PARAM_DEF, NULL,
                 EDT_PARAM_DEF, NULL, EDT_PROP_NONE, NULL_GUID, &stage2bOut);

    // Set up all the links
    // 1a -> 2b
    ocrAddDependence(stage1aOut, stage2bEdtGuid, 0, DB_DEFAULT_MODE); (@ \label{line:outEvtToEdtDep} @)

    // 1b -> 2b
    ocrAddDependence(stage1bOut, stage2bEdtGuid, 1, DB_DEFAULT_MODE);

    // Event satisfied by 2a -> shutdown
    ocrAddDependence(evtGuid, shutdownEdtGuid, 0, DB_DEFAULT_MODE);
    // 2b -> shutdown
    ocrAddDependence(stage2bOut, shutdownEdtGuid, 1, DB_DEFAULT_MODE);

    // Start 1a and 1b
    ocrAddDependence(NULL_GUID, stage1aEdtGuid, 0, DB_DEFAULT_MODE); (@ \label{line:nullGuidToEdt} @)
    ocrAddDependence(NULL_GUID, stage1bEdtGuid, 0, DB_DEFAULT_MODE);

    return NULL_GUID;
}
\end{ocrsnip}
%%%
\subsubsection{Details}
%%%%
%\paragraph{The graph construction}
The snippet of code shows one possible way to construct the irregular
task-graph shown starting on
Line~\ref{line:task-dep-graph}. \texttt{mainEdt} will create {\bf a)}
\texttt{stage1a} and \texttt{stage1b} as they are the next things that
need to execute but also {\bf b)} \texttt{stage2b} and
\texttt{shutdownEdt} because it is the immediate dominator of those
EDTs. In general, it is easiest to create an EDT in its immediate
dominator because that allows any other EDTs who need to feed it
information (necessarily between its dominator and the EDT in
question) to be able to know the value of the opaque GUID created for
hte EDT. \texttt{stage2a}, on the other hand, can be created by
\texttt{stage1a} as no-one else needs to feed information to it.

Most of the ``edges'' in the dependence graph are also created in
\texttt{mainEdt} starting at Line~\ref{line:outEvtToEdtDep}. These
are either between the post-slot (output event) of a source EDT and
an EDT or between a regular event and an EDT. Note also the use of
NULL\_GUID as a source for two dependences starting at
Line~\ref{line:nullGuidToEdt}. A NULL\_GUID as a source for a
dependence immediately satisfies the destination slot; in this case,
it satisfies the unique dependence of \texttt{stage1a} and
\texttt{stage1b} and makes them runable. These two dependences do not
exist in the graph shown starting at Line~\ref{line:task-dep-graph}
but are crucial to avoid a potential race in the program: the output
events of EDTs are similar to ONCE events in the sense that they will
disappear once they are satisfied and therefore, any dependence on
them must be properly setup prior to their potential satisfaction. In
other words, the \texttt{ocrAddDependence} calls starting at
Line~\ref{line:outEvtToEdtDep} must \emph{happen-before} the
satisfaction of \texttt{stage1aOut} and \texttt{stage1bOut}.
%%%%
%\paragraph{Satisfying pre-slots}
This example shows three methods of satisfying an EDT's pre-slots:
\begin{itemize}
\item{Through the use of an explicit dependence array known at EDT
    creation time as shown on Line~\ref{line:expDep};}
\item{Through an output event as shown on
    Line~\ref{line:EdtGuidReturn}. The GUID passed as a return value
    of the EDT function will be passed to the EDT's output event (in
    this case \texttt{stage1aOut}). If the GUID is a data-block's
    GUID, the output event will be satisfied with that data-block. If
    it is an event's GUID, the two events will become linked;}
\item{Through an explicit satisfaction as shown on
    Line~\ref{line:expSatisfy}).}
\end{itemize}

%%%%%
\section{Using a Finish EDT}
%%%%
\subsection{Code example}

The following code demonstrates the use of Finish EDTs by performing a Fast Fourier
Transform on a sparse array of length 256 bytes. For the sake of simplicity, the
array contents and sizes are hardcoded, however, the code can be used as a starting
point for adding more functionality.

\begin{ocrsnip}


/* Example usage of Finish EDT in FFT.
 *
 * Implements the following dependence graph:
 *
 * MainEdt
 *    |
 *
 * FinishEdt
 * {
 *       DFT
 *      /   \
 * FFT-odd FFT-even
 *      \   /
 *     Twiddle
 * }
 *    |
 * Shutdown
 *
 */

#include ``ocr.h''
#include ``math.h''

#define N          256
#define BLOCK_SIZE 16

// The below function performs a twiddle operation on an array x_in
// and places the results in X_real & X_imag. The other arguments
// size and step refer to the size of the array x_in and the offset therein
void ditfft2(double *X_real, double *X_imag, double *x_in, u32 size, u32 step) {
    if(size == 1) {
        X_real[0] = x_in[0];
        X_imag[0] = 0;
    } else {
        ditfft2(X_real, X_imag, x_in, size/2, 2 * step);
        ditfft2(X_real+size/2, X_imag+size/2, x_in+step, size/2, 2 * step);
        u32 k;
        for(k=0;k<size/2;k++) {
            double t_real = X_real[k];
            double t_imag = X_imag[k];
            double twiddle_real = cos(-2 * M_PI * k / size);
            double twiddle_imag = sin(-2 * M_PI * k / size);
            double xr = X_real[k+size/2];
            double xi = X_imag[k+size/2];

            // (a+bi)(c+di) = (ac - bd) + (bc + ad)i
            X_real[k] = t_real +
                (twiddle_real*xr - twiddle_imag*xi);
            X_imag[k] = t_imag +

                (twiddle_imag*xr + twiddle_real*xi);
            X_real[k+size/2] = t_real -
                (twiddle_real*xr - twiddle_imag*xi);
            X_imag[k+size/2] = t_imag -
                (twiddle_imag*xr + twiddle_real*xi);
        }
    }
}

// The below function splits the given array into odd & even portions and
// calls itself recursively via child EDTs that operate on each of the portions,
// till the array operated upon is of size BLOCK_SIZE, a pre-defined
// parameter. It then trivially computes the FFT of this array, then spawns
// twiddle EDTs to combine the results of the children.
ocrGuid_t fftComputeEdt(u32 paramc, u64* paramv, u32 depc, ocrEdtDep_t depv[]) { (@ \label{line:HW_ComputeEdt} @)
    ocrGuid_t computeGuid = paramv[0];
    ocrGuid_t twiddleGuid = paramv[1];
    double *data = (double*)depv[0].ptr;
    ocrGuid_t dataGuid = depv[0].guid;
    u64 size = paramv[2];
    u64 step = paramv[3];
    u64 offset = paramv[4];
    u64 step_offset = paramv[5];
    u64 blockSize = paramv[6];
    double *x_in = (double*)data;
    double *X_real = (double*)(data+offset + size*step);
    double *X_imag = (double*)(data+offset + 2*size*step);

    if(size <= blockSize) {
        ditfft2(X_real, X_imag, x_in+step_offset, size, step);
    } else {
        // DFT even side
        u64 childParamv[7] = { computeGuid, twiddleGuid, size/2, 2 * step,
                               0 + offset, step_offset, blockSize };
        u64 childParamv2[7] = { computeGuid, twiddleGuid, size/2, 2 * step,
                                size/2 + offset, step_offset + step, blockSize };

        ocrGuid_t edtGuid, edtGuid2, twiddleEdtGuid, finishEventGuid, finishEventGuid2;

        ocrEdtCreate(&edtGuid, computeGuid, EDT_PARAM_DEF, childParamv,
                     EDT_PARAM_DEF, NULL, EDT_PROP_FINISH, NULL_GUID,
                     &finishEventGuid); (@ \label{line:HW_FinishEdt1} @)
        ocrEdtCreate(&edtGuid2, computeGuid, EDT_PARAM_DEF, childParamv2,
                     EDT_PARAM_DEF, NULL, EDT_PROP_FINISH, NULL_GUID,
                     &finishEventGuid2); (@ \label{line:HW_FinishEdt2} @)

        ocrGuid_t twiddleDependencies[3] = { dataGuid, finishEventGuid, finishEventGuid2 };
        ocrEdtCreate(&twiddleEdtGuid, twiddleGuid, EDT_PARAM_DEF, paramv, 3,
                     twiddleDependencies, EDT_PROP_FINISH, NULL_GUID, NULL); (@ \label{line:HW_FinishEdt3} @)

        ocrAddDependence(dataGuid, edtGuid, 0, DB_MODE_ITW);
        ocrAddDependence(dataGuid, edtGuid2, 0, DB_MODE_ITW);
    }

    return NULL_GUID;
}

// The below function performs the twiddle operation
ocrGuid_t fftTwiddleEdt(u32 paramc, u64* paramv, u32 depc, ocrEdtDep_t depv[]) { (@ \label{line:HW_TwiddleEdt} @)
    double *data = (double*)depv[0].ptr;
    u64 size = paramv[2];
    u64 step = paramv[3];
    u64 offset = paramv[4];
    double *x_in = (double*)data+offset;
    double *X_real = (double*)(data+offset + size*step);
    double *X_imag = (double*)(data+offset + 2*size*step);

    ditfft2(X_real, X_imag, x_in, size, step);

    return NULL_GUID;
}

ocrGuid_t endEdt(u32 paramc, u64* paramv, u32 depc, ocrEdtDep_t depv[]) { (@ \label{line:HW_EndEdt} @)
    ocrGuid_t dataGuid = paramv[0];

    ocrDbDestroy(dataGuid);
    ocrShutdown();
    return NULL_GUID;
}

ocrGuid_t mainEdt(u32 paramc, u64* paramv, u32 depc, ocrEdtDep_t depv[]) {

    ocrGuid_t computeTempGuid, twiddleTempGuid, endTempGuid;
    ocrEdtTemplateCreate(&computeTempGuid, &fftComputeEdt, 7, 1);
    ocrEdtTemplateCreate(&twiddleTempGuid, &fftTwiddleEdt, 7, 3);
    ocrEdtTemplateCreate(&endTempGuid, &endEdt, 1, 1);
    u32 i;
    double *x;

    ocrGuid_t dataGuid;
    ocrDbCreate(&dataGuid, (void **) &x, sizeof(double) * N * 3, DB_PROP_NONE, NULL_GUID, NO_ALLOC); (@ \label{line:HW_DBCreate} @)

    // Cook up some arbitrary data
    for(i=0;i<N;i++) {
        x[i] = 0;
    }
    x[0] = 1;

    u64 edtParamv[7] = { computeTempGuid, twiddleTempGuid, N, 1, 0, 0, BLOCK_SIZE };
    ocrGuid_t edtGuid, eventGuid, endGuid;

    // Launch compute EDT
    ocrEdtCreate(&edtGuid, computeTempGuid, EDT_PARAM_DEF, edtParamv,
                 EDT_PARAM_DEF, NULL, EDT_PROP_FINISH, NULL_GUID,
                 &eventGuid); (@ \label{line:HW_FinishEdt4} @)

    // Launch finish EDT
    ocrEdtCreate(&endGuid, endTempGuid, EDT_PARAM_DEF, &dataGuid,
                 EDT_PARAM_DEF, NULL, EDT_PROP_FINISH, NULL_GUID,
                 NULL); (@ \label{line:HW_FinishEdt5} @)

    ocrAddDependence(dataGuid, edtGuid, 0, DB_MODE_ITW); (@ \label{line:HW_DBDep} @)
    ocrAddDependence(eventGuid, endGuid, 0, DB_MODE_ITW); (@ \label{line:HW_EventDep} @)

    return NULL_GUID;
}
\end{ocrsnip}
%%%
\subsubsection{Details}

The above code contains a total of 5 functions - a \texttt{mainEdt()} required of all
OCR programs, a \texttt{ditfft2()} that acts as the core of the recursive FFT
computation, calling itself on smaller sizes of the array provided to it, and three
other EDTs that are managed by OCR. They include - \texttt{fftComputeEdt()} in
Line~\ref{line:HW_ComputeEdt} that breaks down the FFT operation on an array into
two FFT operations on the two halves of the array (by spawning two other EDTs of
the same template), as well as an instance of \texttt{fftTwiddleEdt()} shown in
Line~\ref{line:HW_TwiddleEdt} that combines the
results from the two spawned EDTs by applying the FFT ``twiddle'' operation on the
real and imaginary portions of the array. The \texttt{fftComputeEdt()} function stops
spawning EDTs once the size of the array it operates on drops below a pre-defined
\texttt{BLOCK\_SIZE} value. This sets up a recursive cascade of EDTs operating on gradually
smaller data sizes till the \texttt{BLOCK\_SIZE} value is reached, at which point the FFT value
is directly computed, followed by a series of twiddle operations on gradualy larger
data sizes till the entire array has undergone the operation. When this is available,
a final EDT termed \texttt{endEdt()} in Line~\ref{line:HW_EndEdt} is called to
optionally output the value of the
computed FFT, and terminate the program by calling \texttt{ocrShutdown()}. All the
FFT operations are performed on a single datablock created in Line~\ref{line:HW_DBCreate}.
This shortcut is taken for the sake of didactic simplicity. While this is
programmatically correct, a user who desires reducing contention on the single array
may want to break down the datablock into smaller units for each of the EDTs to operate
upon.

For this program to execute correctly, it is apparent that each of the
\texttt{fftTwiddleEdt} instances can not start until all its previous instances have
completed execution. Further, for the sake of program simplicity, an instance of
\texttt{fftComputeEdt}-\texttt{fftTwiddleEdt} pair cannot return until the EDTs that
they spawn have completed execution. The above dependences are enforced using the
concept of \emph{Finish EDTs}. As stated before, a Finish EDT does not return until
all the EDTs spawned by it have completed execution. This simplifies programming,
and does not consume computing resources since a Finish EDT that is not running, is
removed from any computing resources it has used. In this program, no instance of
\texttt{fftComputeEdt} or \texttt{fftTwiddleEdt} returns before the corresponding
EDTs that operates on smaller data sizes have returned, as illustrated in
Lines~\ref{line:HW_FinishEdt1},\ref{line:HW_FinishEdt2} and \ref{line:HW_FinishEdt3}.
Finally, the single \texttt{endEdt()} instance in Line~\ref{line:HW_FinishEdt4} is called
only after all the EDTs spawned by the parent \texttt{fftComputeEdt()}
in Line~\ref{line:HW_FinishEdt5}, return.


%%%%%
\section{Accessing a DataBlock with ``Intent-To-Write'' Mode}
%%%%
This example illustrates the usage model for datablocks accessed with the {\tt Read-Write (RW)} mode.
The {\tt RW} mode ensures that only one master copy of the datablock exists at any time inside a shared address space.
Parallel EDTs can concurrently access a datablock under this mode if they execute inside the same address space.
It is the programmer's responsibility to avoid data races. For example, two parallel EDTs can concurrently update
separate memory regions of the same datablock with the {\tt RW} mode.

\subsection{Code example}
\begin{ocrsnip}
/* Example usage of RW (Read-Write)
 * datablock access mode in OCR
 *
 * Implements the following dependence graph:
 *
 *     mainEdt
 *     [ DB ]
 *      /  \
 * (RW)/    \(RW)
 *    /      \
 * EDT1      EDT2
 *    \      /
 *     [ DB ]
 *   shutdownEdt
 *
 */

#include "ocr.h"

#define N 1000

ocrGuid_t exampleEdt(u32 paramc, u64* paramv, u32 depc, ocrEdtDep_t depv[]) {
    u64 i, lb, ub;
    lb = paramv[0];
    ub = paramv[1];
    u32 *dbPtr = (u32*)depv[0].ptr;

    for (i = lb; i < ub; i++)
        dbPtr[i] += i;

    return NULL_GUID;
}

ocrGuid_t awaitingEdt(u32 paramc, u64* paramv, u32 depc, ocrEdtDep_t depv[]) {
    u64 i;
    PRINTF("Done!\n");
    u32 *dbPtr = (u32*)depv[0].ptr;
    for (i = 0; i < N; i++) {
        if (dbPtr[i] != i * 2)
            break;
    }

    if (i == N) {
        PRINTF("Passed Verification\n");
    } else {
        PRINTF("!!! FAILED !!! Verification\n");
    }

    ocrDbDestroy(depv[0].guid);
    ocrShutdown();
    return NULL_GUID;
}

ocrGuid_t mainEdt(u32 paramc, u64* paramv, u32 depc, ocrEdtDep_t depv[]) {
    u32 i;

    // CHECKER DB
    u32* ptr;
    ocrGuid_t dbGuid;
    ocrDbCreate(&dbGuid, (void**)&ptr, N * sizeof(u32), DB_PROP_NONE, NULL_HINT, NO_ALLOC);
    for(i = 0; i < N; i++)
        ptr[i] = i;
    ocrDbRelease(dbGuid);

    // EDT Template
    ocrGuid_t exampleTemplGuid, exampleEdtGuid1, exampleEdtGuid2, exampleEventGuid1, exampleEventGuid2;
    ocrEdtTemplateCreate(&exampleTemplGuid, exampleEdt, 2 /*paramc*/, 1 /*depc*/);
    u64 args[2];

    // EDT1
    args[0] = 0;
    args[1] = N/2;
    ocrEdtCreate(&exampleEdtGuid1, exampleTemplGuid, EDT_PARAM_DEF, args, EDT_PARAM_DEF, NULL,
        EDT_PROP_NONE, NULL_HINT, &exampleEventGuid1);

    // EDT2
    args[0] = N/2;
    args[1] = N;
    ocrEdtCreate(&exampleEdtGuid2, exampleTemplGuid, EDT_PARAM_DEF, args, EDT_PARAM_DEF, NULL,
        EDT_PROP_NONE, NULL_HINT, &exampleEventGuid2);

    // AWAIT EDT
    ocrGuid_t awaitingTemplGuid, awaitingEdtGuid;
    ocrEdtTemplateCreate(&awaitingTemplGuid, awaitingEdt, 0 /*paramc*/, 3 /*depc*/);
    ocrEdtCreate(&awaitingEdtGuid, awaitingTemplGuid, EDT_PARAM_DEF, NULL, EDT_PARAM_DEF, NULL,
        EDT_PROP_NONE, NULL_HINT, NULL);
    ocrAddDependence(dbGuid,            awaitingEdtGuid, 0, DB_MODE_CONST);
    ocrAddDependence(exampleEventGuid1, awaitingEdtGuid, 1, DB_DEFAULT_MODE);
    ocrAddDependence(exampleEventGuid2, awaitingEdtGuid, 2, DB_DEFAULT_MODE);

    // START
    PRINTF("Start!\n");
    ocrAddDependence(dbGuid, exampleEdtGuid1, 0, DB_MODE_RW);
    ocrAddDependence(dbGuid, exampleEdtGuid2, 0, DB_MODE_RW);

    return NULL_GUID;
}
\end{ocrsnip}

%%%
\subsubsection{Details}
The mainEdt creates a datablock ({\tt dbGuid}) that may be concurrently updated by two children EDTs
({\tt exampleEdtGuid1} and {\tt exampleEdtGuid2}) using the {\tt RW} mode.
{\tt exampleEdtGuid1} and {\tt exampleEdtGuid2} are each created with one dependence on each of them,
while after execution, each of them will satisfy an output event ({\tt exampleEventGuid1} and {\tt exampleEventGuid2}).
The satisfaction of these output events will trigger the execution of an awaiting EDT ({\tt awaitingEdtGuid})
that will verify the correctness of the computation performed by the concurrent EDTs.
{\tt awaitingEdtGuid} has three input dependences.
{\tt dbGuid} is passed into the first input, while the other two would be satisfied by {\tt exampleEventGuid1} and {\tt exampleEventGuid2}.
Once {\tt awaitingEdtGuid}'s dependences have been setup,
the {\tt mainEdt} satisfies the dependences on {\tt exampleEdtGuid1} and {\tt exampleEdtGuid2} with the datablock {\tt dbGuid}.

Both {\tt exampleEdtGuid1} and {\tt exampleEdtGuid2} execute the task function called \textit{exampleEdt}.
This function accesses the contents of the datablock passed in through the dependence slot {\tt 0}.
Based on the parameters passed in, the function updates a range of values on that datablock.
After the datablock has been updated, the EDT returns and in turn satisfies the output event.
Once both EDTs have executed and satisfied their ouput events, the {\tt awaitingEdtGuid} executes function \textit{awaitingEdt}.
This function verifies if the updates done on the datablock by the concurrent EDTs are correct.
Finally, it prints the result of its verification and calls {\tt ocrShutdown}.

%%%%%
\section{Accessing a DataBlock with ``Exclusive-Write'' Mode}
The {\tt Exclusive-Write (EW)} mode allows for an easy implementation
of mutual exclusion of EDTs. When an EDT depends on one or several
data blocks in {\tt EW} mode, the runtime guarantees that only one EDT
in the entire system will have write access to the data block Hence,
the {\tt EW} mode is useful when one wants to guarantee there is no
race condition writing to a data block or when ordering among EDTs
does not matter as long as the execution is in mutual exclusion. The
following example shows how two EDTs may share access to a data block
in {\tt RW} mode, while one EDT requires {\tt EW} access. In this
situation the programmer cannot assume the order in which the EDTs are
executed. It might be that EDT1 and EDT2 are executed simultaneously
or independently, while EDT3 happens either before, after or in
between the others.

%%%%

\subsection{Code example}
\begin{ocrsnip}
/* Example usage of EW (Exclusive-Write)
 * data block access mode in OCR
 *
 * Implements the following dependence graph:
 *
 *       mainEdt
 *       [ DB ]
 *      / |     \
 * (RW)/  |(RW)  \(EW)
 *    /   |       \
 * EDT1  EDT2    EDT3
 *    \   |      /
 *     \  |     /
 *      \ |    /
 *       [ DB ]
 *     shutdownEdt
 *
 */

#include "ocr.h"

#define NB_ELEM_DB 20

ocrGuid_t shutdownEdt(u32 paramc, u64* paramv, u32 depc, ocrEdtDep_t depv[]) {
    u64 * data = (u64 *) depv[3].ptr;
    u32 i = 0;
    while (i < NB_ELEM_DB) {
        ocrPrintf("%"PRId32" ",data[i]);
        i++;
    }
    ocrPrintf("\n");
    ocrDbDestroy(depv[3].guid);
    ocrShutdown();
    return NULL_GUID;
}

ocrGuid_t writerEdt(u32 paramc, u64* paramv, u32 depc, ocrEdtDep_t depv[]) {
    u64 * data = (u64 *) depv[0].ptr;
    u64 lb = paramv[0];
    u64 ub = paramv[1];
    u64 value = paramv[2];
    u32 i = lb;
    while (i < ub) {
        data[i] += value;
        i++;
    }
    return NULL_GUID;
}

ocrGuid_t mainEdt(u32 paramc, u64* paramv, u32 depc, ocrEdtDep_t depv[]) {
    void * dbPtr;
    ocrGuid_t dbGuid;
    u32 nbElem = NB_ELEM_DB;
    ocrDbCreate(&dbGuid, &dbPtr, sizeof(u64) * NB_ELEM_DB, 0, NULL_HINT, NO_ALLOC);
    u64 i = 0;
    int * data = (int *) dbPtr;
    while (i < nbElem) {
        data[i] = 0;
        i++;
    }
    ocrDbRelease(dbGuid);

    ocrGuid_t shutdownEdtTemplateGuid;
    ocrEdtTemplateCreate(&shutdownEdtTemplateGuid, shutdownEdt, 0, 4);
    ocrGuid_t shutdownGuid;
    ocrEdtCreate(&shutdownGuid, shutdownEdtTemplateGuid, 0, NULL, EDT_PARAM_DEF, NULL,
                 EDT_PROP_NONE, NULL_HINT, NULL);
    ocrAddDependence(dbGuid, shutdownGuid, 3, DB_MODE_CONST);

    ocrGuid_t writeEdtTemplateGuid;
    ocrEdtTemplateCreate(&writeEdtTemplateGuid, writerEdt, 3, 2);

    ocrGuid_t eventStartGuid;
    ocrEventCreate(&eventStartGuid, OCR_EVENT_ONCE_T, false);

    // RW '1' from 0 to N/2 (potentially concurrent with writer 1, but different range)
    ocrGuid_t oeWriter0Guid;
    ocrGuid_t writer0Guid;
    u64 writerParamv0[3] = {0, NB_ELEM_DB/2, 1};
    ocrEdtCreate(&writer0Guid, writeEdtTemplateGuid, EDT_PARAM_DEF, writerParamv0, EDT_PARAM_DEF, NULL,
                 EDT_PROP_NONE, NULL_HINT, &oeWriter0Guid);
    ocrAddDependence(oeWriter0Guid, shutdownGuid, 0, false);
    ocrAddDependence(dbGuid, writer0Guid, 0, DB_MODE_RW);
    ocrAddDependence(eventStartGuid, writer0Guid, 1, DB_MODE_CONST);

    // RW '2' from N/2 to N (potentially concurrent with writer 0, but different range)
    ocrGuid_t oeWriter1Guid;
    ocrGuid_t writer1Guid;
    u64 writerParamv1[3] = {NB_ELEM_DB/2, NB_ELEM_DB, 2};
    ocrEdtCreate(&writer1Guid, writeEdtTemplateGuid, EDT_PARAM_DEF, writerParamv1, EDT_PARAM_DEF, NULL,
                 EDT_PROP_NONE, NULL_HINT, &oeWriter1Guid);
    ocrAddDependence(oeWriter1Guid, shutdownGuid, 1, false);
    ocrAddDependence(dbGuid, writer1Guid, 0, DB_MODE_RW);
    ocrAddDependence(eventStartGuid, writer1Guid, 1, DB_MODE_CONST);

    // EW '3' from N/4 to 3N/4
    ocrGuid_t oeWriter2Guid;
    ocrGuid_t writer2Guid;
    u64 writerParamv2[3] = {NB_ELEM_DB/4, (NB_ELEM_DB/4)*3, 3};
    ocrEdtCreate(&writer2Guid, writeEdtTemplateGuid, EDT_PARAM_DEF, writerParamv2, EDT_PARAM_DEF, NULL,
                 EDT_PROP_NONE, NULL_HINT, &oeWriter2Guid);
    ocrAddDependence(oeWriter2Guid, shutdownGuid, 2, false);
    ocrAddDependence(dbGuid, writer2Guid, 0, DB_MODE_EW);
    ocrAddDependence(eventStartGuid, writer2Guid, 1, DB_MODE_CONST);

    ocrEventSatisfy(eventStartGuid, NULL_GUID);

    return NULL_GUID;
}
\end{ocrsnip}
%%%

%%%%%
\section{Acquiring contents of a DataBlock as a dependence input}
This example illustrates the usage model for accessing the contents of a data block.
The data contents of a data block are made available to the EDT through the input slots in depv.
The input slots contain two fields: the GUID of the data block and pointer to the contents of the data block.
The runtime process grabs a pointer to the contents through a process called ``acquire''.
The acquires of all data blocks accessed inside the EDT have to happen
before the EDT starts execution. This implies that the runtime requires knowledge of
which data blocks it needs to acquire. That information is given to the runtime through
the process of dependence satisfaction. As a result, a data block's contents are available
to the EDT only if that data block has been passed in as the input on a dependence slot or
if the data block is created inside the EDT.
%%%%
\subsection{Code example}
\begin{ocrsnip}
/* Example to show how DB guids can be passed through another DB.
 * Note: DB contents can be accessed by an EDT only when they arrive
 * in a dependence slot.
 *
 * Implements the following dependence graph:
 *
 *     mainEdt
 *     [ DB1 ]
 *        |
 *       EDT1
 *        |
 *     [ DB0 ]
 *   shutdownEdt
 *
 */

#include "ocr.h"

#define VAL 42

ocrGuid_t exampleEdt(u32 paramc, u64* paramv, u32 depc, ocrEdtDep_t depv[]) {
    ocrGuid_t *dbPtr = (ocrGuid_t*)depv[0].ptr;
    ocrGuid_t passedDb = dbPtr[0];
    ocrPrintf("Passing DB: "GUIDF"\n", GUIDA(passedDb));
    ocrDbDestroy(depv[0].guid);
    return passedDb;
}

ocrGuid_t awaitingEdt(u32 paramc, u64* paramv, u32 depc, ocrEdtDep_t depv[]) {
    u64 i;
    u32 *dbPtr = (u32*)depv[0].ptr;
    ocrPrintf("Received: %"PRIu32"\n", dbPtr[0]);
    ocrDbDestroy(depv[0].guid);
    ocrShutdown();
    return NULL_GUID;
}

ocrGuid_t mainEdt(u32 paramc, u64* paramv, u32 depc, ocrEdtDep_t depv[]) {
    u32 i;

    // Create DBs
    u32* ptr0;
    ocrGuid_t* ptr1;
    ocrGuid_t db0Guid, db1Guid;
    ocrDbCreate(&db0Guid, (void**)&ptr0, sizeof(u32), DB_PROP_NONE, NULL_HINT, NO_ALLOC);
    ocrDbCreate(&db1Guid, (void**)&ptr1, sizeof(ocrGuid_t), DB_PROP_NONE, NULL_HINT, NO_ALLOC);
    ptr0[0] = VAL;
    ptr1[0] = db0Guid;
    ocrPrintf("Sending: %"PRIu32" in DB: "GUIDF"\n", ptr0[0], GUIDA(db0Guid));
    ocrDbRelease(db0Guid);
    ocrDbRelease(db1Guid);

    // Create Middle EDT
    ocrGuid_t exampleTemplGuid, exampleEdtGuid, exampleEventGuid;
    ocrEdtTemplateCreate(&exampleTemplGuid, exampleEdt, 0 /*paramc*/, 1 /*depc*/);
    ocrEdtCreate(&exampleEdtGuid, exampleTemplGuid, EDT_PARAM_DEF, NULL, EDT_PARAM_DEF, NULL,
        EDT_PROP_NONE, NULL_HINT, &exampleEventGuid);

    // Create AWAIT EDT
    ocrGuid_t awaitingTemplGuid, awaitingEdtGuid;
    ocrEdtTemplateCreate(&awaitingTemplGuid, awaitingEdt, 0 /*paramc*/, 1 /*depc*/);
    ocrEdtCreate(&awaitingEdtGuid, awaitingTemplGuid, EDT_PARAM_DEF, NULL, EDT_PARAM_DEF, NULL,
        EDT_PROP_NONE, NULL_HINT, NULL);
    ocrAddDependence(exampleEventGuid, awaitingEdtGuid, 0, DB_DEFAULT_MODE);

    // START Middle EDT
    ocrAddDependence(db1Guid, exampleEdtGuid, 0, DB_DEFAULT_MODE);

    return NULL_GUID;
}
\end{ocrsnip}
%%%
\subsubsection{Details}

The mainEdt creates two data blocks ({\tt db0Guid} and {\tt db1Guid}).
It then sets the content of {\tt db0Guid} to be an user-define value,
while the content of {\tt db1Guid} is set to be the GUID value of {\tt db0Guid}.
The runtime then creates an EDT ({\tt exampleEdtGuid}) that takes one input dependence.
It creates another EDT ({\tt awaitingEdtGuid}) and makes it dependent on the satisfaction of the {\tt exampleEdtGuid}'s output event ({\tt exampleEventGuid}).
Finally, mainEdt satisfies the dependence of {\tt exampleEdtGuid} with the data block {\tt db1Guid}.

Once {\tt exampleEdtGuid} starts executing function ``exampleEdt'', the contents of {\tt db1Guid} are read.
The function then retrieves the GUID of the data block {\tt db0Guid} from the contents of {\tt db1Guid}.
Now in order to read the contents of {\tt db0Guid}, the function satisfies the output event with {\tt db0Guid}.

Inside the final EDT function ``awaitingEdt'', the contents of {\tt db0Guid} can be read.
The function prints the content read from the data block and finally calls ``ocrShutdown''.

