% This is an included file. See the master file for more information.
%

\chapter{OCR Change  History}
\label{chap:OCR Change History}
\label{chap:Appendix D}
\begin{description}
\item[September 2014] Release of OCR 0.9 including the first version
  of this specification.
\item[April 2015] Release of OCR 0.95.  Fixed some typos in the spec and
cleaned up some subtle flaws in the memory model.
\item[June 2015] Release of OCR 1.0.0. Restructured the specification for clarity and
updated the API to make the names of the memory modes more intuitive. Also moved
the API documentation from doxygen to human-readable TEX with proper specification
language.
\item[September 2015] Release of OCR 1.0.1. Added Section B on labeled GUIDs and
hints. Minor other clarifications.
\item[March 2016] Release of OCR 1.1.0 and 1.0.2. Notable changes include:
\begin{itemize}
\item{API changes for \code{ocrDbCreate} and \code{ocrEdtCreate} to
  use hints instead of affinities}
\item{Added GUID management section (\ref{sec:OCRGuidManagement})}
\item{Added counted and channel event extensions}
\item{Clarification on API return codes, multiple acquire of the same
  data block}
\item{Added section on version numbers}
\end{itemize}
\item[December 2016] Release of OCR 1.2.0 (Candidate). This release breaks certain
APIs thus the naming change to 1.2. This is a candidate release as no implementations
fully implement 1.2.0 as of December 2016. Notable changes include:
\begin{itemize}
\item{Added the ``downgrade release'' of data blocks}
\item{Added the possibility to use hints when creating events}
\item{Added the possibility to self-identify an EDT (and its output event)}
\item{Added the possibility of specifying an output event when creating an EDT}
\item{Made the API calls more ``asynchronous'' friendly in particular by: {\bf a)} removing the
option of using EDT\_PARAM\_DEF, and {\bf b)} relaxing the requirements on the values returned
through error codes.}
\item{Added several clarifications, in particular on: {\bf a)} OCR object lifetimes,
{\bf b)} zero-sized data blocks, {\bf c)} behavior of EDTs destroyed in a finish scope, and
{\bf d)} the flags during event creation.}
\item{Clarified the behavior of several extensions, namely: {\bf a)} hints, {\bf b)} labeled GUIDs,
and {\bf c)} the ELS.}
\end{itemize}
\end{description}

% This is the end of app-D-history
