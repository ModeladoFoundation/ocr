% This is an included file. See the master file for more information.
%

\chapter{OCR Implementation Details}
\label{chap:OCR Implementation Detail}
\label{chap:Appendix C}

\section{Data structures}

Every OCR object is identified by a Globally Unique ID or GUID.  This is an 
opaque object to programmers working with OCR with type ocrGUID\_t.  This
is defined internal to OCR as:

 typedef intptr\_t ocrGuid\_t

macros
\#define DB\_ACCESS\_MODE\_MASK 0x1E Runtime reserved constant to support 
work wtih data blocks

\section{Event management}

Events in OCR make use of an enumeration type ocrEventTypes\_t.
In addition to the API defined values defined in the OCR core specification,
one additional ennumerated value is defined internal to OCR
\begin{itemize}

\item OCR\_EVENT\_T\_MAX

\end{itemize}

\section{EDT management}

For some reason (which will be documented here) the OCR team decided that they needed
to implement the \code{ocrEDTTemplateCreate} function as a macro.  Therefore they 
needed to define the following inside OCR:

%
% EdtTemplateCreate_internal
%
\hypertarget{group__OCREDT_gaa481a8e8f1a2af9aba623cb01b01bc1b}{\index{Event Driven Task A\-P\-I@{Event Driven Task A\-P\-I}!ocr\-Edt\-Template\-Create\-\_\-internal@{ocr\-Edt\-Template\-Create\-\_\-internal}}
\index{ocr\-Edt\-Template\-Create\-\_\-internal@{ocr\-Edt\-Template\-Create\-\_\-internal}!Event Driven Task API@{Event Driven Task A\-P\-I}}
\subsubsection[{ocr\-Edt\-Template\-Create\-\_\-internal}]{\setlength{\rightskip}{0pt plus 5cm}{\bf u8} ocr\-Edt\-Template\-Create\-\_\-internal (
\begin{DoxyParamCaption}
\item[{{\bf ocr\-Guid\-\_\-t} $\ast$}]{guid, }
\item[{{\bf ocr\-Edt\-\_\-t}}]{func\-Ptr, }
\item[{{\bf u32}}]{paramc, }
\item[{{\bf u32}}]{depc, }
\item[{const char $\ast$}]{func\-Name}
\end{DoxyParamCaption}
)}}\label{group__OCREDT_gaa481a8e8f1a2af9aba623cb01b01bc1b}


Creates an E\-D\-T template. 

An E\-D\-T template encapsulates the E\-D\-T function and, optionally, the number of parameters and arguments that E\-D\-Ts instanciated from this template will use. It needs to be created only once for each function that will serve as an E\-D\-T; reusing the same template from multiple E\-D\-Ts of the same type may improve performance as this allows the runtime to collect information about multiple instances of the same type of E\-D\-T.


\begin{DoxyParams}[1]{Parameters}
\mbox{\tt out}  & {\em guid} & Runtime created G\-U\-I\-D for the newly created E\-D\-T Template \\
\hline
\mbox{\tt in}  & {\em func\-Ptr} & E\-D\-T function. This function must be of type \hyperlink{group__OCRTypesEDT_ga6d853145262e968a06fbe8a6799bef64}{ocr\-Edt\-\_\-t}. \\
\hline
\mbox{\tt in}  & {\em paramc} & Number of parameters that the E\-D\-Ts will take. If not known or variable, this can be E\-D\-T\-\_\-\-P\-A\-R\-A\-M\-\_\-\-U\-N\-K \\
\hline
\mbox{\tt in}  & {\em depc} & Number of pre-\/slots that the E\-D\-Ts will have. If not known or variable, this can be E\-D\-T\-\_\-\-P\-A\-R\-A\-M\-\_\-\-U\-N\-K \\
\hline
\mbox{\tt in}  & {\em func\-Name} & User-\/specified name for the template (used in debugging)\\
\hline
\end{DoxyParams}
\begin{DoxyReturn}{Returns}
a status code\-:
\begin{DoxyItemize}
\item 0\-: successful
\item E\-N\-O\-M\-E\-M\-: No memory available to allocate the template
\end{DoxyItemize}
\end{DoxyReturn}
\begin{DoxyNote}{Note}
You should use \hyperlink{group__OCREDT_ga9649098ffad669c1c820cd821792d67a}{ocr\-Edt\-Template\-Create()} as opposed to this internal function. If O\-C\-R\-\_\-\-E\-N\-A\-B\-L\-E\-\_\-\-E\-D\-T\-\_\-\-N\-A\-M\-I\-N\-G is enabled, the C name of the function will be used as func\-Name. 
\end{DoxyNote}





\section{OCR interface to a limited subset of C standard library functionality}

OCR must support a wide variety of platforms including simulators that emmulate real systems.  
Often, core functionality provided by standard C librarires are not availabel on all platforms
and as a result, an OCR program cannot depend on these functions.  Two functions supported 
as macros have been included in the internal, development versions of OCR.

\begin{itemize}

\item \#define ASSERT(a)

\item  \#define VERIFY(cond, format,...)

\end{itemize}

These macros and the api PRINT capability are supported internally by the following fucntions:

\begin{itemize}

\item void \_ocrAssert ( bool val, const char  file, u32 line )

\end{itemize}

% This is the end of app-B-history
