This example illustrates the creation of a fork-join pattern in OCR.

%%%%
\subsection{Code example}
\begin{ocrsnip}
/* Example of a "fork-join" pattern in OCR
 *
 * Implements the following dependence graph:
 *
 *   mainEdt
 *   /    \
 * fun1   fun2
 *   \    /
 * shutdownEdt
 *
 */

#include "ocr.h" (@ \label{line:FJ_include} @)

ocrGuid_t fun1(u32 paramc, u64* paramv, u32 depc, ocrEdtDep_t depv[]) {
    int* k;
    ocrGuid_t db_guid;
    ocrDbCreate(&db_guid,(void **) &k, sizeof(int), 0, NULL_HINT, NO_ALLOC); (@ \label{line:FJ_db1}@)
    k[0]=1; (@ \label{line:FJ_k1} @)
    PRINTF("Hello from fun1, sending k = %"PRId32"\n",*k);
    return db_guid; (@ \label{line:FJ_retDB1} @)
}

ocrGuid_t fun2(u32 paramc, u64* paramv, u32 depc, ocrEdtDep_t depv[]) {
    int* k;
    ocrGuid_t db_guid;
    ocrDbCreate(&db_guid,(void **) &k, sizeof(int), 0, NULL_HINT, NO_ALLOC); (@ \label{line:FJ_db2}@)
    k[0]=2; (@ \label{line:FJ_k2} @)
    PRINTF("Hello from fun2, sending k = %"PRId32"\n",*k);
    return db_guid; (@ \label{line:FJ_retDB2} @)
}

ocrGuid_t shutdownEdt(u32 paramc, u64* paramv, u32 depc, ocrEdtDep_t depv[]) { (@ \label{line:FJ_shutdown}@)
    PRINTF("Hello from shutdownEdt\n");
    int* data1 = (int*) depv[0].ptr;
    int* data2 = (int*) depv[1].ptr;
    PRINTF("Received data1 = %"PRId32", data2 = %"PRId32"\n", *data1, *data2);
    ocrDbDestroy(depv[0].guid);
    ocrDbDestroy(depv[1].guid);
    ocrShutdown();
    return NULL_GUID;
}

ocrGuid_t mainEdt(u32 paramc, u64* paramv, u32 depc, ocrEdtDep_t depv[]) { (@ \label{line:FJ_mainEdt} @)
    PRINTF("Starting mainEdt\n");
    ocrGuid_t edt1_template, edt2_template, edt3_template;
    ocrGuid_t edt1, edt2, edt3, outputEvent1, outputEvent2;

    //Create templates for the EDTs
    ocrEdtTemplateCreate(&edt1_template, fun1, 0, 1); (@ \label{line:FJ_edtTemplt1} @)
    ocrEdtTemplateCreate(&edt2_template, fun2, 0, 1); (@ \label{line:FJ_edtTemplt2} @)
    ocrEdtTemplateCreate(&edt3_template, shutdownEdt, 0, 2); (@ \label{line:FJ_edtTemplt3} @)

    //Create the EDTs
    ocrEdtCreate(&edt1, edt1_template, EDT_PARAM_DEF, NULL, EDT_PARAM_DEF, NULL, EDT_PROP_NONE, NULL_HINT, &outputEvent1); (@ \label{line:FJ_edt1} @)
    ocrEdtCreate(&edt2, edt2_template, EDT_PARAM_DEF, NULL, EDT_PARAM_DEF, NULL, EDT_PROP_NONE, NULL_HINT, &outputEvent2); (@ \label{line:FJ_edt2} @)
    ocrEdtCreate(&edt3, edt3_template, EDT_PARAM_DEF, NULL, 2, NULL, EDT_PROP_NONE, NULL_HINT, NULL); (@ \label{line:FJ_edt3} @)

    //Setup dependences for the shutdown EDT
    ocrAddDependence(outputEvent1, edt3, 0, DB_MODE_CONST); (@ \label{line:FJ_edt3Dep1} @)
    ocrAddDependence(outputEvent2, edt3, 1, DB_MODE_CONST); (@ \label{line:FJ_edt3Dep2} @)

    //Start execution of the parallel EDTs
    ocrAddDependence(NULL_GUID, edt1, 0, DB_DEFAULT_MODE); (@ \label{line:FJ_edt1start} @)
    ocrAddDependence(NULL_GUID, edt2, 0, DB_DEFAULT_MODE); (@ \label{line:FJ_edt2start} @)
    return NULL_GUID;
}
\end{ocrsnip}
%%%
\subsubsection{Details}

%%%%
The \texttt{ocr.h} file included on Line~\ref{line:FJ_include}
contains all of the main OCR APIs. The \texttt{mainEdt} is
shown on Line~\ref{line:FJ_mainEdt}. It is called by the runtime
as a \texttt{main} function is not provided
(more details in \texttt{hello.c}).

The \texttt{mainEdt} creates three templates (Lines~\ref{line:FJ_edtTemplt1}, ~\ref{line:FJ_edtTemplt2}
and ~\ref{line:FJ_edtTemplt3}), respectively for three different EDTs
(Lines~\ref{line:FJ_edt1}, ~\ref{line:FJ_edt2} and ~\ref{line:FJ_edt3}).
An EDT is created as an instance of an EDT template.
This template stores metadata about EDT,
optionally defines the number of dependences and parameters used when
creating an instance of an EDT, and is a container for the function
that will be executed by an EDT. This function is called the EDT function.
For the EDTs, \texttt{edt1}, \texttt{edt2} and \texttt{edt3}, the EDT functions
are, \texttt{fun1}, \texttt{fun2} and \texttt{shutdownEdt}, respectively.
The last parameter to \texttt{ocrEdtTemplateCreate} is the total number of
data blocks on which the EDTs depends. The signature of EDT creation API,
\texttt{ocrEdtCreate}, is shown in Lines~\ref{line:FJ_edt1}, ~\ref{line:FJ_edt2}
and ~\ref{line:FJ_edt3}. When \texttt{edt1} and \texttt{edt2} will complete,
they will satisfy the output events \texttt{outputEvent1} and \texttt{outputEvent2} repectively.
This is not required for \texttt{edt3}. However, \texttt{edt3} should execute only
when the events \texttt{outputEvent1} and \texttt{outputEvent2} are satisfied.
This is done by setting up dependencies on \texttt{edt3} by using the API \texttt{ocrAddDependence},
as shown in Lines~\ref{line:FJ_edt3Dep1} and ~\ref{line:FJ_edt3Dep2}.
This spawns \texttt{edt3} but it will not execute until both the events
are satisfied. Finally, the EDTs \texttt{edt1} and \texttt{edt2} are
spawned in Lines~\ref{line:FJ_edt1start} and ~\ref{line:FJ_edt2start} respectively.
As they do not have any dependencies, they execute the associated EDT functions
in parallel. These functions (\texttt{fun1} and \texttt{fun2}) creates
data blocks using the API \texttt{ocrDbCreate} (Lines~\ref{line:FJ_db1} and ~\ref{line:FJ_db2}).
The data is written to the data blocks and the GUID is returned (Lines~\ref{line:FJ_retDB1}
and ~\ref{line:FJ_retDB2}). This will satisfy the events on which the \texttt{edt3}
is waiting. The EDT function \texttt{shutdownEdt} executes
and calls \texttt{ocrShutdown} after reading and destroying the two data blocks.

%%%%
