%%%%
This example illustrates the usage model for data blocks accessed with
the {\tt Read-Write (RW)} mode. The {\tt RW} mode ensures that only
one master copy of the data block exists at any time inside a shared
address space. Parallel EDTs can concurrently access a data block
under this mode if they execute inside the same address space. It is
the programmer's responsibility to avoid data races. For example, two
parallel EDTs can concurrently update separate memory regions of the
same data block with the {\tt RW} mode.

\subsection{Code example}
\begin{ocrsnip}
/* Example usage of RW (Read-Write)
 * data block access mode in OCR
 *
 * Implements the following dependence graph:
 *
 *     mainEdt
 *     [ DB ]
 *      /  \
 * (RW)/    \(RW)
 *    /      \
 * EDT1      EDT2
 *    \      /
 *     [ DB ]
 *   shutdownEdt
 *
 */

#include "ocr.h"

#define N 1000

ocrGuid_t exampleEdt(u32 paramc, u64* paramv, u32 depc, ocrEdtDep_t depv[]) {
    u64 i, lb, ub;
    lb = paramv[0];
    ub = paramv[1];
    u32 *dbPtr = (u32*)depv[0].ptr;

    for (i = lb; i < ub; i++)
        dbPtr[i] += i;

    return NULL_GUID;
}

ocrGuid_t awaitingEdt(u32 paramc, u64* paramv, u32 depc, ocrEdtDep_t depv[]) {
    u64 i;
    ocrPrintf("Done!\n");
    u32 *dbPtr = (u32*)depv[0].ptr;
    for (i = 0; i < N; i++) {
        if (dbPtr[i] != i * 2)
            break;
    }

    if (i == N) {
        ocrPrintf("Passed Verification\n");
    } else {
        ocrPrintf("!!! FAILED !!! Verification\n");
    }

    ocrDbDestroy(depv[0].guid);
    ocrShutdown();
    return NULL_GUID;
}

ocrGuid_t mainEdt(u32 paramc, u64* paramv, u32 depc, ocrEdtDep_t depv[]) {
    u32 i;

    // CHECKER DB
    u32* ptr;
    ocrGuid_t dbGuid;
    ocrDbCreate(&dbGuid, (void**)&ptr, N * sizeof(u32), DB_PROP_NONE, NULL_HINT, NO_ALLOC);
    for(i = 0; i < N; i++)
        ptr[i] = i;
    ocrDbRelease(dbGuid);

    // EDT Template
    ocrGuid_t exampleTemplGuid, exampleEdtGuid1, exampleEdtGuid2, exampleEventGuid1, exampleEventGuid2;
    ocrEdtTemplateCreate(&exampleTemplGuid, exampleEdt, 2 /*paramc*/, 1 /*depc*/);
    u64 args[2];

    // EDT1
    args[0] = 0;
    args[1] = N/2;
    ocrEdtCreate(&exampleEdtGuid1, exampleTemplGuid, EDT_PARAM_DEF, args, EDT_PARAM_DEF, NULL,
        EDT_PROP_NONE, NULL_HINT, &exampleEventGuid1);

    // EDT2
    args[0] = N/2;
    args[1] = N;
    ocrEdtCreate(&exampleEdtGuid2, exampleTemplGuid, EDT_PARAM_DEF, args, EDT_PARAM_DEF, NULL,
        EDT_PROP_NONE, NULL_HINT, &exampleEventGuid2);

    // AWAIT EDT
    ocrGuid_t awaitingTemplGuid, awaitingEdtGuid;
    ocrEdtTemplateCreate(&awaitingTemplGuid, awaitingEdt, 0 /*paramc*/, 3 /*depc*/);
    ocrEdtCreate(&awaitingEdtGuid, awaitingTemplGuid, EDT_PARAM_DEF, NULL, EDT_PARAM_DEF, NULL,
        EDT_PROP_NONE, NULL_HINT, NULL);
    ocrAddDependence(dbGuid,            awaitingEdtGuid, 0, DB_MODE_CONST);
    ocrAddDependence(exampleEventGuid1, awaitingEdtGuid, 1, DB_DEFAULT_MODE);
    ocrAddDependence(exampleEventGuid2, awaitingEdtGuid, 2, DB_DEFAULT_MODE);

    // START
    ocrPrintf("Start!\n");
    ocrAddDependence(dbGuid, exampleEdtGuid1, 0, DB_MODE_RW);
    ocrAddDependence(dbGuid, exampleEdtGuid2, 0, DB_MODE_RW);

    return NULL_GUID;
}
\end{ocrsnip}

%%%
\subsubsection{Details}
The mainEdt creates a data block ({\tt dbGuid}) that may be
concurrently updated by two children EDTs ({\tt exampleEdtGuid1} and
{\tt exampleEdtGuid2}) using the {\tt RW} mode. {\tt exampleEdtGuid1}
and {\tt exampleEdtGuid2} are each created with one dependence,
while after execution, each of them will satisfy an output
event ({\tt exampleEventGuid1} and {\tt exampleEventGuid2}). The
satisfaction of these output events will trigger the execution of an
awaiting EDT ({\tt awaitingEdtGuid}) that will verify the correctness
of the computation performed by the concurrent EDTs. {\tt
awaitingEdtGuid} has three input dependences: {\tt dbGuid} is passed
into the first input, while the other two would be satisfied by {\tt
exampleEventGuid1} and {\tt exampleEventGuid2}. Once {\tt
awaitingEdtGuid}'s dependences have been setup, the {\tt mainEdt}
satisfies the dependences on {\tt exampleEdtGuid1} and {\tt
exampleEdtGuid2} with the data block {\tt dbGuid}.

Both {\tt exampleEdtGuid1} and {\tt exampleEdtGuid2} execute the task
function called \textit{exampleEdt}. This function accesses the
contents of the data block passed in through the dependence slot {\tt
0}. Based on the parameters passed in, the function updates a range
of values on that data block. After the data block has been updated,
the EDT returns and in turn satisfies the output event. Once both
EDTs have executed and satisfied their ouput events, the {\tt
awaitingEdtGuid} executes function \textit{awaitingEdt}. This
function verifies if the updates done on the data block by the
concurrent EDTs are correct. Finally, it prints the result of its
verification and calls {\tt ocrShutdown}.
