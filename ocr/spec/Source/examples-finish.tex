%%%%
\subsection{Code example}

The following code demonstrates the use of Finish EDTs by performing a Fast Fourier
Transform on a sparse array of length 256 bytes. For the sake of simplicity, the
array contents and sizes are hardcoded, however, the code can be used as a starting
point for adding more functionality.

\begin{ocrsnip}


/* Example usage of Finish EDT in FFT.
 *
 * Implements the following dependence graph:
 *
 * MainEdt
 *    |
 *
 * FinishEdt
 * {
 *       DFT
 *      /   \
 * FFT-odd FFT-even
 *      \   /
 *     Twiddle
 * }
 *    |
 * Shutdown
 *
 */

#include ``ocr.h''
#include ``math.h''

#define N          256
#define BLOCK_SIZE 16

// The below function performs a twiddle operation on an array x_in
// and places the results in X_real & X_imag. The other arguments
// size and step refer to the size of the array x_in and the offset therein
void ditfft2(double *X_real, double *X_imag, double *x_in, u32 size, u32 step) {
    if(size == 1) {
        X_real[0] = x_in[0];
        X_imag[0] = 0;
    } else {
        ditfft2(X_real, X_imag, x_in, size/2, 2 * step);
        ditfft2(X_real+size/2, X_imag+size/2, x_in+step, size/2, 2 * step);
        u32 k;
        for(k=0;k<size/2;k++) {
            double t_real = X_real[k];
            double t_imag = X_imag[k];
            double twiddle_real = cos(-2 * M_PI * k / size);
            double twiddle_imag = sin(-2 * M_PI * k / size);
            double xr = X_real[k+size/2];
            double xi = X_imag[k+size/2];

            // (a+bi)(c+di) = (ac - bd) + (bc + ad)i
            X_real[k] = t_real +
                (twiddle_real*xr - twiddle_imag*xi);
            X_imag[k] = t_imag +

                (twiddle_imag*xr + twiddle_real*xi);
            X_real[k+size/2] = t_real -
                (twiddle_real*xr - twiddle_imag*xi);
            X_imag[k+size/2] = t_imag -
                (twiddle_imag*xr + twiddle_real*xi);
        }
    }
}

// The below function splits the given array into odd & even portions and
// calls itself recursively via child EDTs that operate on each of the portions,
// till the array operated upon is of size BLOCK_SIZE, a pre-defined
// parameter. It then trivially computes the FFT of this array, then spawns
// twiddle EDTs to combine the results of the children.
ocrGuid_t fftComputeEdt(u32 paramc, u64* paramv, u32 depc, ocrEdtDep_t depv[]) { (@ \label{line:HW_ComputeEdt} @)
    ocrGuid_t computeGuid = paramv[0];
    ocrGuid_t twiddleGuid = paramv[1];
    double *data = (double*)depv[0].ptr;
    ocrGuid_t dataGuid = depv[0].guid;
    u64 size = paramv[2];
    u64 step = paramv[3];
    u64 offset = paramv[4];
    u64 step_offset = paramv[5];
    u64 blockSize = paramv[6];
    double *x_in = (double*)data;
    double *X_real = (double*)(data+offset + size*step);
    double *X_imag = (double*)(data+offset + 2*size*step);

    if(size <= blockSize) {
        ditfft2(X_real, X_imag, x_in+step_offset, size, step);
    } else {
        // DFT even side
        u64 childParamv[7] = { computeGuid, twiddleGuid, size/2, 2 * step,
                               0 + offset, step_offset, blockSize };
        u64 childParamv2[7] = { computeGuid, twiddleGuid, size/2, 2 * step,
                                size/2 + offset, step_offset + step, blockSize };

        ocrGuid_t edtGuid, edtGuid2, twiddleEdtGuid, finishEventGuid, finishEventGuid2;

        ocrEdtCreate(&edtGuid, computeGuid, EDT_PARAM_DEF, childParamv,
                     EDT_PARAM_DEF, NULL, EDT_PROP_FINISH, NULL_GUID,
                     &finishEventGuid); (@ \label{line:HW_FinishEdt1} @)
        ocrEdtCreate(&edtGuid2, computeGuid, EDT_PARAM_DEF, childParamv2,
                     EDT_PARAM_DEF, NULL, EDT_PROP_FINISH, NULL_GUID,
                     &finishEventGuid2); (@ \label{line:HW_FinishEdt2} @)

        ocrGuid_t twiddleDependencies[3] = { dataGuid, finishEventGuid, finishEventGuid2 };
        ocrEdtCreate(&twiddleEdtGuid, twiddleGuid, EDT_PARAM_DEF, paramv, 3,
                     twiddleDependencies, EDT_PROP_FINISH, NULL_GUID, NULL); (@ \label{line:HW_FinishEdt3} @)

        ocrAddDependence(dataGuid, edtGuid, 0, DB_MODE_RW);
        ocrAddDependence(dataGuid, edtGuid2, 0, DB_MODE_RW);
    }

    return NULL_GUID;
}

// The below function performs the twiddle operation
ocrGuid_t fftTwiddleEdt(u32 paramc, u64* paramv, u32 depc, ocrEdtDep_t depv[]) { (@ \label{line:HW_TwiddleEdt} @)
    double *data = (double*)depv[0].ptr;
    u64 size = paramv[2];
    u64 step = paramv[3];
    u64 offset = paramv[4];
    double *x_in = (double*)data+offset;
    double *X_real = (double*)(data+offset + size*step);
    double *X_imag = (double*)(data+offset + 2*size*step);

    ditfft2(X_real, X_imag, x_in, size, step);

    return NULL_GUID;
}

ocrGuid_t endEdt(u32 paramc, u64* paramv, u32 depc, ocrEdtDep_t depv[]) { (@ \label{line:HW_EndEdt} @)
    ocrGuid_t dataGuid = paramv[0];

    ocrDbDestroy(dataGuid);
    ocrShutdown();
    return NULL_GUID;
}

ocrGuid_t mainEdt(u32 paramc, u64* paramv, u32 depc, ocrEdtDep_t depv[]) {

    ocrGuid_t computeTempGuid, twiddleTempGuid, endTempGuid;
    ocrEdtTemplateCreate(&computeTempGuid, &fftComputeEdt, 7, 1);
    ocrEdtTemplateCreate(&twiddleTempGuid, &fftTwiddleEdt, 7, 3);
    ocrEdtTemplateCreate(&endTempGuid, &endEdt, 1, 1);
    u32 i;
    double *x;

    ocrGuid_t dataGuid;
    ocrDbCreate(&dataGuid, (void **) &x, sizeof(double) * N * 3, DB_PROP_NONE, NULL_GUID, NO_ALLOC); (@ \label{line:HW_DBCreate} @)

    // Cook up some arbitrary data
    for(i=0;i<N;i++) {
        x[i] = 0;
    }
    x[0] = 1;

    u64 edtParamv[7] = { computeTempGuid, twiddleTempGuid, N, 1, 0, 0, BLOCK_SIZE };
    ocrGuid_t edtGuid, eventGuid, endGuid;

    // Launch compute EDT
    ocrEdtCreate(&edtGuid, computeTempGuid, EDT_PARAM_DEF, edtParamv,
                 EDT_PARAM_DEF, NULL, EDT_PROP_FINISH, NULL_GUID,
                 &eventGuid); (@ \label{line:HW_FinishEdt4} @)

    // Launch finish EDT
    ocrEdtCreate(&endGuid, endTempGuid, EDT_PARAM_DEF, &dataGuid,
                 EDT_PARAM_DEF, NULL, EDT_PROP_FINISH, NULL_GUID,
                 NULL); (@ \label{line:HW_FinishEdt5} @)

    ocrAddDependence(dataGuid, edtGuid, 0, DB_MODE_RW); (@ \label{line:HW_DBDep} @)
    ocrAddDependence(eventGuid, endGuid, 0, DB_MODE_RW); (@ \label{line:HW_EventDep} @)

    return NULL_GUID;
}
\end{ocrsnip}
%%%
\subsubsection{Details}

The above code contains a total of 5 functions - a \texttt{mainEdt()} required of all
OCR programs, a \texttt{ditfft2()} that acts as the core of the recursive FFT
computation, calling itself on smaller sizes of the array provided to it, and three
other EDTs that are managed by OCR. They include - \texttt{fftComputeEdt()} in
Line~\ref{line:HW_ComputeEdt} that breaks down the FFT operation on an array into
two FFT operations on the two halves of the array (by spawning two other EDTs of
the same template), as well as an instance of \texttt{fftTwiddleEdt()} shown in
Line~\ref{line:HW_TwiddleEdt} that combines the
results from the two spawned EDTs by applying the FFT ``twiddle'' operation on the
real and imaginary portions of the array. The \texttt{fftComputeEdt()} function stops
spawning EDTs once the size of the array it operates on drops below a pre-defined
\texttt{BLOCK\_SIZE} value. This sets up a recursive cascade of EDTs operating on gradually
smaller data sizes till the \texttt{BLOCK\_SIZE} value is reached, at which point the FFT value
is directly computed, followed by a series of twiddle operations on gradualy larger
data sizes till the entire array has undergone the operation. When this is available,
a final EDT termed \texttt{endEdt()} in Line~\ref{line:HW_EndEdt} is called to
optionally output the value of the
computed FFT, and terminate the program by calling \texttt{ocrShutdown()}. All the
FFT operations are performed on a single datablock created in Line~\ref{line:HW_DBCreate}.
This shortcut is taken for the sake of didactic simplicity. While this is
programmatically correct, a user who desires reducing contention on the single array
may want to break down the datablock into smaller units for each of the EDTs to operate
upon.

For this program to execute correctly, it is apparent that each of the
\texttt{fftTwiddleEdt} instances can not start until all its previous instances have
completed execution. Further, for the sake of program simplicity, an instance of
\texttt{fftComputeEdt}-\texttt{fftTwiddleEdt} pair cannot return until the EDTs that
they spawn have completed execution. The above dependences are enforced using the
concept of \emph{Finish EDTs}. As stated before, a Finish EDT does not return until
all the EDTs spawned by it have completed execution. This simplifies programming,
and does not consume computing resources since a Finish EDT that is not running, is
removed from any computing resources it has used. In this program, no instance of
\texttt{fftComputeEdt} or \texttt{fftTwiddleEdt} returns before the corresponding
EDTs that operates on smaller data sizes have returned, as illustrated in
Lines~\ref{line:HW_FinishEdt1},\ref{line:HW_FinishEdt2} and \ref{line:HW_FinishEdt3}.
Finally, the single \texttt{endEdt()} instance in Line~\ref{line:HW_FinishEdt4} is called
only after all the EDTs spawned by the parent \texttt{fftComputeEdt()}
in Line~\ref{line:HW_FinishEdt5}, return.

