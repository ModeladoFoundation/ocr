This example illustrates the most basic OCR program: a single function
that prints the message ``Hello World!'' on the screen and exits.
%%%%
\subsection{Code example}
The following code will print the string ``Hello World!'' to the
standard output and exit. Note that this program is fully functional
(ie: there is no need for a \texttt{main} function).

\begin{ocrsnip}
#include <ocr.h> (@ \label{line:HW_include} @)

ocrGuid_t mainEdt(u32 paramc, u64* paramv, u32 depc, ocrEdtDep_t depv[]) { (@ \label{line:HW_mainEdt} @)
    PRINTF('Hello World!\n'); (@ \label{line:HW_printf}@)
    ocrShutdown(); (@ \label{line:HW_shutdown}@)
    return NULL_GUID;
}
\end{ocrsnip}
%%%
\subsubsection{Details}
The \texttt{ocr.h} file included on Line~\ref{line:HW_include}
contains all of the main OCR APIs. Other more experimental or extended
APIs are also located in the \texttt{extensions/} folder of the
include directory.

EDT's signature is shown on Line~\ref{line:HW_mainEdt}. A special EDT,
named \texttt{mainEdt} is called by the runtime if the programmer does not
provide a \texttt{main} function\footnote{Note that if the programmer
  \emph{does} provide a \texttt{main} function, it is the
  responsability of the programmer to properly initialize the runtime,
  call the first EDT to execute and properly shutdown the
  runtime. Refer to the legacy mode extension and the
  \texttt{ocr-legacy.h} header file for more detail.}.

The \texttt{ocrShutdown} function called on
Line~\ref{line:HW_shutdown} should be called once and only once by all
OCR programs to indicate that the program has terminated. The runtime
will then shutdown and any non-executed EDTs at that time are not
guaranteed to execute.
