The {\tt Exclusive-Write (EW)} mode allows for an easy implementation
of mutual exclusion of EDTs. When an EDT depends on one or several
data blocks in {\tt EW} mode, the runtime guarantees that only one EDT
in the entire system will have write access to the data block Hence,
the {\tt EW} mode is useful when one wants to guarantee there is no
race condition writing to a data block or when ordering among EDTs
does not matter as long as the execution is in mutual exclusion. The
following example shows how two EDTs may share access to a data block
in {\tt RW} mode, while one EDT requires {\tt EW} access. In this
situation the programmer cannot assume the order in which the EDTs are
executed. It might be that EDT1 and EDT2 are executed simultaneously
or independently, while EDT3 happens either before, after or in
between the others.

%%%%

\subsection{Code example}
\begin{ocrsnip}
/* Example usage of EW (Exclusive-Write)
 * data block access mode in OCR
 *
 * Implements the following dependence graph:
 *
 *       mainEdt
 *       [ DB ]
 *      / |     \
 * (RW)/  |(RW)  \(EW)
 *    /   |       \
 * EDT1  EDT2    EDT3
 *    \   |      /
 *     \  |     /
 *      \ |    /
 *       [ DB ]
 *     shutdownEdt
 *
 */

#include "ocr.h"

#define NB_ELEM_DB 20

ocrGuid_t shutdownEdt(u32 paramc, u64* paramv, u32 depc, ocrEdtDep_t depv[]) {
    u64 * data = (u64 *) depv[3].ptr;
    u32 i = 0;
    while (i < NB_ELEM_DB) {
        PRINTF("%"PRId32" ",data[i]);
        i++;
    }
    PRINTF("\n");
    ocrDbDestroy(depv[3].guid);
    ocrShutdown();
    return NULL_GUID;
}

ocrGuid_t writerEdt(u32 paramc, u64* paramv, u32 depc, ocrEdtDep_t depv[]) {
    u64 * data = (u64 *) depv[0].ptr;
    u64 lb = paramv[0];
    u64 ub = paramv[1];
    u64 value = paramv[2];
    u32 i = lb;
    while (i < ub) {
        data[i] += value;
        i++;
    }
    return NULL_GUID;
}

ocrGuid_t mainEdt(u32 paramc, u64* paramv, u32 depc, ocrEdtDep_t depv[]) {
    void * dbPtr;
    ocrGuid_t dbGuid;
    u32 nbElem = NB_ELEM_DB;
    ocrDbCreate(&dbGuid, &dbPtr, sizeof(u64) * NB_ELEM_DB, 0, NULL_HINT, NO_ALLOC);
    u64 i = 0;
    int * data = (int *) dbPtr;
    while (i < nbElem) {
        data[i] = 0;
        i++;
    }
    ocrDbRelease(dbGuid);

    ocrGuid_t shutdownEdtTemplateGuid;
    ocrEdtTemplateCreate(&shutdownEdtTemplateGuid, shutdownEdt, 0, 4);
    ocrGuid_t shutdownGuid;
    ocrEdtCreate(&shutdownGuid, shutdownEdtTemplateGuid, 0, NULL, EDT_PARAM_DEF, NULL,
                 EDT_PROP_NONE, NULL_HINT, NULL);
    ocrAddDependence(dbGuid, shutdownGuid, 3, DB_MODE_CONST);

    ocrGuid_t writeEdtTemplateGuid;
    ocrEdtTemplateCreate(&writeEdtTemplateGuid, writerEdt, 3, 2);

    ocrGuid_t eventStartGuid;
    ocrEventCreate(&eventStartGuid, OCR_EVENT_ONCE_T, false);

    // RW '1' from 0 to N/2 (potentially concurrent with writer 1, but different range)
    ocrGuid_t oeWriter0Guid;
    ocrGuid_t writer0Guid;
    u64 writerParamv0[3] = {0, NB_ELEM_DB/2, 1};
    ocrEdtCreate(&writer0Guid, writeEdtTemplateGuid, EDT_PARAM_DEF, writerParamv0, EDT_PARAM_DEF, NULL,
                 EDT_PROP_NONE, NULL_HINT, &oeWriter0Guid);
    ocrAddDependence(oeWriter0Guid, shutdownGuid, 0, false);
    ocrAddDependence(dbGuid, writer0Guid, 0, DB_MODE_RW);
    ocrAddDependence(eventStartGuid, writer0Guid, 1, DB_MODE_CONST);

    // RW '2' from N/2 to N (potentially concurrent with writer 0, but different range)
    ocrGuid_t oeWriter1Guid;
    ocrGuid_t writer1Guid;
    u64 writerParamv1[3] = {NB_ELEM_DB/2, NB_ELEM_DB, 2};
    ocrEdtCreate(&writer1Guid, writeEdtTemplateGuid, EDT_PARAM_DEF, writerParamv1, EDT_PARAM_DEF, NULL,
                 EDT_PROP_NONE, NULL_HINT, &oeWriter1Guid);
    ocrAddDependence(oeWriter1Guid, shutdownGuid, 1, false);
    ocrAddDependence(dbGuid, writer1Guid, 0, DB_MODE_RW);
    ocrAddDependence(eventStartGuid, writer1Guid, 1, DB_MODE_CONST);

    // EW '3' from N/4 to 3N/4
    ocrGuid_t oeWriter2Guid;
    ocrGuid_t writer2Guid;
    u64 writerParamv2[3] = {NB_ELEM_DB/4, (NB_ELEM_DB/4)*3, 3};
    ocrEdtCreate(&writer2Guid, writeEdtTemplateGuid, EDT_PARAM_DEF, writerParamv2, EDT_PARAM_DEF, NULL,
                 EDT_PROP_NONE, NULL_HINT, &oeWriter2Guid);
    ocrAddDependence(oeWriter2Guid, shutdownGuid, 2, false);
    ocrAddDependence(dbGuid, writer2Guid, 0, DB_MODE_EW);
    ocrAddDependence(eventStartGuid, writer2Guid, 1, DB_MODE_CONST);

    ocrEventSatisfy(eventStartGuid, NULL_GUID);

    return NULL_GUID;
}
\end{ocrsnip}
%%%
