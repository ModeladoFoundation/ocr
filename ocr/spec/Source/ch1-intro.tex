% This is an included file. See the master file for more information.
%

\chapter{Introduction}
\label{chap:introduction}
Extreme scale computers (such as proposed Exascale computers) contain
so many components that the aggregate mean-time-between-failure is small
compared to the runtime of an application. Programming models (and
supporting compilers and runtime systems) must therefore support a
variety of features unique to these machines:
\begin{itemize}
\item The ability for a programmer to
express O(billion) concurrency in an application program.

\item The ability of a computation to make progress towards a useful
result even as components within the system fail.

\item The ability of a computation to dynamically adapt to a high
degree of variability in the performance and energy consumption of
system components to support efficient execution.

\item The ability to either hide overheads behind useful computation
or have overheads small enough to allow applications to exhibit strong
scaling across the entire exascale system.

\end{itemize}

There are a number of research projects in progress to develop a runtime system
for extreme scale computers. This specification describes one of
these research runtime systems: the \emph{Open Community Runtime} or \emph{OCR}.

The fundamental idea behind OCR is to consider a computation as a
dynamically generated directed acyclic graph (DAG) of tasks operating on
relocatable blocks of data (Data Blocks). Task execution is managed by events;
when the data blocks and events a task depends upon are satisfied, the
task is ready and will at some later point run on the system.
Representing a computation in terms of this event-driven DAG of tasks
decouples the work of a computation
from the ``Units of execution'' that carry out the computation.  The work
of a computation is virtualized so it can be relocated within a scalable computer to
optimize execution of the parallel program. More importantly,
virtualization allows a program to respond to hardware issues by
relocating work from failed components onto working
components~\cite{Vrvilo14}.

OCR provides a global name space that holds references to tasks,
events and data blocks\footnote{Data blocks, described in
Section~\ref{sec:datablocks} are exposed through a global ID which
provides a reference to the data block itself. A task can access the
contents of data blocks either {\bf a)} created inside the task and
{\bf b)} data blocks whose IDs are listed as requisite input to the
task before it is scheduled for execution.}, which supports the transparent
scheduling and relocation of these data across hardware resources.

A DAG of Event driven tasks can be used to support a wide range of execution models,
including data-flow (when events are associated with data-blocks),
fork-join (when events enable the execution of post-join
continuations), bulk-synchronous processing (when event trees can be
used to build scalable barriers and collective operations), and
combinations thereof.

\section{Scope}
\label{sec:Scope}

OCR is a vehicle to support research on programming models and
runtime systems for extreme scale
computers~\cite{ExascaleSoftwareStudy2009,SaHS10}. This specification
defines the state of OCR at a fixed point in its development. There
are several limitations in OCR that will be relaxed as it continues to
develop.

OCR is a runtime system and collection of low level Application
Programming Interfaces (APIs). While some programmers will directly
work with the APIs defined by OCR, the most common use of OCR will be
to support higher level programming models. Therefore, OCR lacks high
level constructs familiar to traditional parallel programmers such as
\code{reductions} and \code{parallel for}\footnote{Reductions can be
supported in OCR using an accumulator/reducer
approach~\cite{Frigo:2009:ROC:1583991.1584017,SCZS13} and
\code{parallel for} can be supported in OCR
using a fork-join decomposition similar to the cilk\_for construct.}.

All parallelism must be specified explicitly in OCR. It does not
extract the concurrency in a program on behalf of a programmer. The
OCR execution model is a low level model; abstract enough to support
relocation of tasks and data to support resiliency or to minimize the
energy consumed by a computation, but low level enough to cleanly map
onto the hardware of extreme scale computers.

OCR is designed to handle dynamic task driven algorithms expressed in
terms of a directed acyclic graph (DAG). In an OCR DAG, each node is
visited only once. This makes irregular problems based on dynamic
graphs easier to express. However, it means that OCR may be less
effective for regular problems that benefit from static load balancing
or for problems that depend on iteration over regular control
structures.
%
% Vivek wanted to put a comment here about the use of annotations to support
% regular algorithms. I opted to not include this since at this time, we don't have
% such annotations in OCR. When we add them and show that the address this problem,
% we can update this text accordingly.

OCR is defined in terms of a C library. Programs written in any
language that includes an interface to C should be able to work with
OCR.

OCR tasks are expressed as event driven tasks (EDTs).
The overheads associated with OCR API calls depend on the underlying system software and
hardware in an extreme scale platform.  On current systems, the overhead of
creating and scheduling an event driven task can be fairly heavy-weight.
On system with hardware support for task queues, the
overheads can be significantly lower.  An OCR programmer should experiment with their
implementation of OCR to understand the overheads associated with managing EDTs and
assure that the work per EDT is great enough to offset OCR overheads.

OCR is currently a research runtime system, developed as an
open-source community project. It does not as yet have the level of
investment needed to develop a production system that can be used for
serious application deployment.
% This is the end of ch1-intro.tex of the OCR specification.
