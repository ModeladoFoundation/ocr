% This is an included file. See the master file for more information.
%
\chapter{Introduction}
\label{chap:introduction}
Extreme scale computers (such as proposed Exascale computers) contain
so many components that the aggregate mean-time-between-failure (MTBF)
is small
compared to the runtime of an application. Programming models (and
supporting compilers and runtime systems) must therefore support a
variety of features unique to these machines:
\begin{itemize}
\item The ability for a programmer to
express O(billion) concurrency in an application program.

\item The ability of a computation to make progress towards a useful
result even as components within the system fail.

\item The ability of a computation to dynamically adapt to a high
degree of variability in the performance and energy consumption of
system components to support efficient execution.

\item The ability to either hide overheads behind useful computation
or have overheads small enough to allow applications to exhibit strong
scaling across the entire exascale system.

\end{itemize}

There are a number of active research projects to develop runtime systems
for extreme scale computers. This specification describes one of
these research runtime systems: the \emph{Open Community Runtime} or \emph{OCR}.

The fundamental idea behind OCR is to consider a computation as a
dynamically generated directed acyclic graph (DAG) \index{DAG}~\cite{
TaSa11,Tasirlar11,Zuckerman:2011:UCP:2000417.2000424} of tasks
operating on relocatable blocks of data (which we call data blocks in
OCR). Task execution is synchronized through the use of events.
When the data blocks and events a task depends upon are satisfied, the
preconditions for the execution of the task~\cite{SSWS13} are met and the
task will, at some later point, run on the system. OCR tasks are
\emph{non-blocking}\index{Non-blocking}. This means that once all
preconditions of a task have been met, the task will eventually run to completion
regardless of the behavior of any other tasks or events.

Representing a computation in terms of an event-driven DAG of tasks
decouples the work of a computation
from the ``units of execution'' that carry out the computation. The work
of a computation is virtualized giving OCR the flexibility to relocate tasks
to respond to failures in the system~\cite{Vrvilo14}, achieve a better balance of load
among the processing elements of the computer, or to optimize memory
and energy consumption~\cite{GZCS10,Guo10,CTBCCGYS13,SbBS14}.

Representing the data in terms of data blocks similarly
decouples the data in a computation from the computer's memory subsystem.
This supports transparent placement and dynamic migration of data
across hardware resources.
%%%%%
\section{Scope}
\label{sec:Scope}
OCR is a vehicle to support research on programming models and
runtime systems for extreme scale
computers~\cite{ExascaleSoftwareStudy2009,SaHS10}. This specification
defines the state of OCR at a fixed point in its development and will
continuously evolve as limitations are identified and addressed.

OCR is both a low level runtime system designed to map onto a wide range of scalable
computer systems as well as a collection of low level application programming
interfaces (API).
It provides the capabilities needed to support a wide range of programming models
including data-flow (when events are associated with data blocks),
fork-join (when events enable the execution of post-join
continuations), bulk-synchronous processing (when event trees can be
used to build scalable barriers and collective operations), and
combinations thereof. While some programmers will directly
work with the APIs defined by OCR, the most common use of OCR will be
to support higher level programming models. Therefore, OCR lacks high
level constructs familiar to traditional parallel programmers such as
\code{reductions} and \code{parallel for}\footnote{Reductions can be
supported in OCR using an accumulator/reducer
approach~\cite{Frigo:2009:ROC:1583991.1584017,SCZS13} and
\code{parallel for} can be supported in OCR
using a fork-join decomposition similar to the cilk\_for construct.}.
OCR is therefore not designed to be the primary interface for application
level programming.

All parallelism must be specified explicitly in OCR; OCR does not
extract the concurrency in a program on behalf of a programmer nor does
it make any implicit assumption about the ordering of tasks in contrast to,
for example, Legion\cite{BauerLegion2012} which executes tasks in parallel
as long as the appearance of the sequential order specified implicitly by
the ordering of the code is maintained.

OCR is designed to handle dynamic task driven algorithms expressed in
terms of a directed acyclic graph (DAG). In an OCR DAG, each node is
visited only once. This makes irregular problems based on dynamic
graphs easier to express. However, it means that OCR may be less
effective for regular problems that benefit from static load balancing
or for problems that depend on iteration over regular control
structures.

OCR is defined in terms of a C library. Programs written in any
language that includes an interface to C should be able to work with
OCR.

OCR tasks are expressed as event driven tasks (EDTs).
The overheads associated with OCR API calls depend on the underlying
system software and hardware. On current systems, the overhead of
creating and scheduling an event driven task can be fairly high.
On system with hardware support for task queues, the
overheads can be significantly lower. An OCR programmer should experiment with their
implementation of OCR to understand the overheads associated with managing EDTs and
assure that the work per EDT is great enough to offset OCR overheads.

OCR is currently a research runtime system, developed as an
open-source community project. It does not as yet have the level of
investment needed to develop a production system that can be used for
serious application deployment.
%%%%%
\section{Version numbers}
As OCR is evolving, newer versions of the API may break
backward compatibility with older versions of the API. To help the
programmer navigate these incompatibilities, the version number for
the OCR API will follow the following rules:
\begin{itemize}
\item{An OCR version number is composed of three integers $X$, $Y$ and
  $Z$ and represented as $X.Y.Z$.}
\item{API compatibility is guaranteed between any versions that only
  vary on the last digit. For example, $1.0.1$ is compatible with
  $1.0.5$.}
\item{A change in any of the other digits indicates that an API
  incompatibility exists. The magnitude of this incompatibility is
  indicated by whether the middle digit changed (smaller change) or
  the first digit changed (big change).}
\item{OCR provides macros to check the API version implemented as
  well as any extensions that are supported.}
\end{itemize}
% This is the end of ch1-intro.tex of the OCR specification.
